\documentclass[12pt]{amsart}
\setlength{\parskip}{.1in}
\setlength{\parindent}{0cm}
%myalterations
\usepackage{amssymb}
\usepackage{amsmath}
\usepackage[usenames,dvipsnames,svgnames,table]{xcolor}
\usepackage[colorlinks=true,urlcolor=blue,pdfborder={0 0 .5}pdfnewwindow=true]{hyperref}
\usepackage{enumitem}
%\usepackage{amsthm}
\usepackage{graphicx}
\usepackage{verbatim}
\usepackage{tabularx}
%\usepackage{arydshln,leftidx,mathtools}
\usepackage{bm}
\usepackage{tikz}
\usepackage{tikz-cd}
\usepackage{hyperref}
\usepackage{bm}
\usepackage{stmaryrd}

%\setlength{\dashlinedash}{.4pt}
%\setlength{\dashlinegap}{.8pt}
%\usepackage{amsthm}
\usepackage{verbatim}
%\usepackage{commath}
%My commands
%environment abbreviations
\newcommand{\benu}{\begin{enumerate}}
\newcommand{\eenu}{\end{enumerate}}
\newcommand{\bed}{\begin{description}}
\newcommand{\ed}{\end{description}}
\theoremstyle{definition}
\newtheorem{theorem}{Theorem}
\newtheorem{notation}[theorem]{Notation}
\newtheorem{exercise}[theorem]{Exercise}
\newcommand{\bex}{\begin{exercise}}
\newcommand{\ex}{\end{exercise}}

\newcommand{\pru}{{ \bfseries \textcolor{red}{Proof:} }}

%symbol definitions
\newcommand{\un}[1]{\underline{#1}}
\newcommand{\mbZ}{\mathbb{Z}}
\newcommand{\mbR}{\mathbb{R}}
\newcommand{\mbN}{\mathbb{N}}
\newcommand{\mbQ}{\mathbb{Q}}
\newcommand{\mbC}{\mathbb{C}}
\newcommand{\mbF}{\mathbb{F}}
\newcommand{\mcS}{\mathcal{S}}
\newcommand{\mcP}{\mathcal{P}}
\newcommand{\mcU}{\mathcal{U}}
\newcommand{\mcV}{\mathcal{V}}

\newcommand{\hra}{\hookrightarrow}
\newcommand{\tra}{\twoheadrightarrow}
\newcommand{\lra}{\leftrightarrow}
\newcommand{\ep}{\epsilon}
\newcommand{\Ra}{\Rightarrow}
\newcommand{\mb}[1]{\mathbb{#1}}
\newcommand{\mc}[1]{\mathcal{#1}}
\newcommand{\bfs}[1]{{\bfseries #1}}
\newcommand{\bs}[1]{\boldsymbol{#1}}
%Operator definitions
\DeclareMathOperator{\Irr}{Irr}
\DeclareMathOperator{\triv}{triv}
\DeclareMathOperator{\cyc}{cyc}
\DeclareMathOperator{\lcm}{lcm}
\DeclareMathOperator{\expo}{x}
\DeclareMathOperator{\ord}{o}
\DeclareMathOperator{\imm}{im}
\DeclareMathOperator{\sgn}{sgn}
\DeclareMathOperator{\Sym}{Sym}
\DeclareMathOperator{\alt}{alt}
\DeclareMathOperator{\irr}{irr}
\DeclareMathOperator{\eqt}{Equiv}
\DeclareMathOperator{\pat}{Part}
%\DeclareMathOperator{\sgn}{sgn}
%\DeclareMathOperator{\Aut}{Aut}
\DeclareMathOperator{\Gl}{Gl}
\DeclareMathOperator{\M}{M}
\DeclareMathOperator{\Id}{Id}
\DeclareMathOperator{\fixx}{Fix}
\DeclareMathOperator{\suppp}{Supp}
\DeclareMathOperator{\gl}{Gl}
\DeclareMathOperator{\id}{Id}
\DeclareMathOperator{\Aut}{Aut}
\DeclareMathOperator{\Inn}{Inn}
\DeclareMathOperator{\orb}{orb}
\DeclareMathOperator{\ii}{I}
\DeclareMathOperator{\im}{im}
\DeclareMathOperator{\Fix}{Fix}
\DeclareMathOperator{\Co}{Co}
\DeclareMathOperator{\md}{md}
\DeclareMathOperator{\qt}{qt}
\DeclareMathOperator{\ExtendedGCD}{ExtendedGCD}
\DeclareMathOperator{\Mod}{Mod}
\DeclareMathOperator{\GCD}{GCD}
\newcommand{\nms}{\negmedspace}
\newcommand{\nts}{\negthinspace}

\newcommand{\itep}{\item {\bfseries Problem}\ }
\newcommand{\gen}[1]{\langle \nts#1 \nts\rangle}
\newcommand{\quot}[2]{#1/ #2}
\newcommand{\order}[1]{\left|<\nts #1 \nts s>\right|}

%These next two commands are for making answers. 
\newcommand{\beans}{\begin{description} \item[{ \bfseries \textcolor{red}{Answer}}]\ }
\newcommand{\eans }{\end{description}}
%\newcommand{\begin{comment}ex}{{ \bfseries \textcolor{red}{Answer}}}

%To turn the answer into problem sets use replace to replace \begin{comment} with \begin{comment} and \\end{comment}  by \end{comment}.
\newcommand{\lieb}[3][{{}}]{\frac{d^#1 #2}{d\,#3^#1}}

\title{\textbf{Math 521 - Problem Set 1}}
\author{Guy Matz}
\date{\today}

\begin{document} 

%\maketitle
%\newpage % Q1

\begin{enumerate}[series=p]
\itep 44 (85)
Suppose that $M$ is compact and that $\mcU$ is an open covering of $M$ which is "redundant" in the sense that each $p \in M$ is contained in at least two members of $\mcU$.  Show that $\mcU$ reduces to a finite subcovering with the same proprrty.
\newpage

\itep 45 (86)
Suppose that every open covering of $M$ has a positive Lebesgue number.  Give an example of such an $M$ that is not compact.
\newpage

\itep 46 (87)
Give a direct proof that $[a,b]$ is covering compact. [Hint: let $\mcU$ be an open covering of $[a,b]$ and consider the set:
$$C = \{x \in [a,b]: \text{finitely many members of  } \mcU \text{ cover }[a,x] \}$$
\newpage

\itep 47 (88)
Give a direct proof that a closed subset $A$ of a covering compact set $K$ is covering compact.
\newpage

\itep 48 (89)
Give a proof of Theorem 39 using open coverings.  That is, assume that $A$ is a covering compact subset of $M$ and $f: M \to N$ is continuous.  Prove directly that $fA$ is covering compact. [Hint: What is the criterion for continuity in terms of pre-images?]
\newpage

\itep 49 (90)
Suppose that $f: M \to N$ is a continuous bijection and $M$ is covering compact.  Prove directly that $f$ is a homeomorphism.
\newpage

\itep 50 (91)
Suppose that $M$ is covering compact and that $f:M \to N$ is continuous.  Use the Lebesgue number lemma to prove that $f$ is uniformly continuous.  [Hint: Consider the covering of $N$ by $\epsilon/2$ neighborhoods $\{N_{\epsilon/2}(q): q \in N\}$ and its pre-image in $M, \{f^{pre}(N_{\epsilon/2}(q)) :q \in N \}$]
\newpage

\itep 51 (92)
Give a direct proof that the nested decreasing intersection of non-empty covering compact sets is non-empty.  [Hint: If $A_1 \supset A_2 \supset \dots$ are covering compact, consider the open sets $U_n = A^c_n$.  If $\bigcap A_n = \emptyset$, what does $\{U_N \}$ cover?] 
\newpage

\itep 52 (93)

Generalize Exercise 51 as follows.  Suppose that $M$ is covering compact and $\mathcal{C}$ is a collection of closed subsets of $M$ such that every intersection of finitely many members of $\mathcal{C}$ is non-empty.  (Such a collection $\mathcal{C}$ is said to have the \textbf{finite intersection property}.)  Prove that the \textbf{grand intersection} $\bigcap_{C \in \mathcal{C}}C$ is non-empty.  [Hint: Consider the collection of open sets $\mcU = \{C^c : C \in \mathcal{C}\}$]
\newpage

\itep 53 (94)

If every collection of closed subsets of $M$ which has the finite intersection property also has a non-empty grand intersection, prove that $M$ is covering compact.  [Hint: Given an open covering $\mcU = \{U_\alpha\}]$. consider the collection of closed set $\mc{C} = \{U_\alpha^c\}$.]
\newpage

\itep 109(a-d) (99)

Let $M, N$ be non-empty metric spaces and $P = M \times N$
	\benu
		\item If $M,N$ are perfect, prove that $P$ is perfect
		\item If $M, N$ are totally disconnected, prove that $P$ is totally disconnected.
		\item What are the converses?
		\item Infer that the Cartesian product of Cantor spaces is a Cantor space.  (We already know that the Cartesian product of compacts is compact.)
	\eenu

\newpage

\itep 118 (103)

Prove that there is a continuous surjection  $\mbR \to \mbR^2$.  What about $\mbR^m$
\newpage

\itep 89 (122)

Recall that $p$ is a \textbf{cluster point} of $S$ if each $M_rp$ contains indefinitely many points of $S$.  The set of cluster points of $S$ is denoted as $S'$.  Prove:
	\benu
		\item If $S \subset T$ then $S' \subset T'$
		\item $(S \subset T)' = S' \cup T'$
		\item $S' = (\overline{S})'$
		\item $S'$ is closed in $M$; that is $S'' \subset S'$ where $S'' = (S')'$
		\item Calculate $\mbN', \mbQ', \mbR', (\mbR \setminus \mbQ)', \mbQ''$
		\item Let $T$ be the set of points $\{1/n : n \in \mbN\}$.  Calculate $T'$ and $T''$
	\eenu
\newpage

\itep 91 (124)

Recall that $p$ is an interior point $S \subset M$ if some $M_rp$ is contained in $S$.  The set of interior points of $S$ is the \textbf{interior} of $S$ and is denoted int S.  For all subsets $S, T$ of the metric space $M$ prove:
	\benu
		\item int $S = S \setminus \partial S$
		\item int $S = (\overline{S^c})^c$
		\item int(int $S$) = int $S$
		\item int($S \cap T$) = int $S$ $\cap$ int $T$
		\item What are the dual equations for the closure
		\item Prove that int($S \cup T$) $\supset$ (int $S$ $\cup$ int $T$).  Show by example that the inclusion can be strict, i.e., not an equality.
	\eenu
\newpage

\itep 92 (125)

A point $p$ is a boundary point of a set $S \supset M$ if every neighborhood $M_rp$ contains points of both $S$ and $S^c$.  The \textbf{boundary} of $S$ is denoted $\partial S$.  For all subsets $S, T$ of a metric space $M$ prove:
	\benu
		\item $S$ is clopen iff $\partial S = \emptyset$
		\item $\partial S = \partial S^c$
		\item $\partial \partial S \subset \partial S$
		\item $\partial \partial \partial S \subset \partial \partial S$
		\item $\partial (S cap T) \subset \partial S \cup \partial T$
		\item Give an example in which (c) is a strict inclusion, $\partial \partial \neq \partial S$
		\item What about (e)?
	\eenu
\newpage


\itep 130 (152)

Write jingles at least as good as crap.
\begin{verbatim}
( Q: What's the contour integral around )
( Western Europe? A: Zero, because all  )
( the Poles are in Eastern Europe!      )
(                                       )
( Addendum: Actually, there ARE some    )
( Poles in Western Europe, but they     )
( are removable!                        )
(                                       )
( Q: An English mathematician (I forgot )
( who) was asked by his                 )
( very religious colleague: Do you      )
( believe in one God? A: Yes, up to     )
( isomorphism!                          )
(                                       )
( Q: What is a compact city? A: It's a  )
( city that can be guarded by finitely  )
( many near-sighted policemen!          )
(                                       )
( -- Peter Lax                          )
---------------------------------------
o
 o
  ___  
{~._.~}
 ( Y )
()~*~()   
(_)-(_)   
\end{verbatim}
\newpage

\end{enumerate}
\end{document}
