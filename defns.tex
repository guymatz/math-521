\documentclass{article}
\usepackage{color,amsmath,amssymb}
\usepackage{framed}
\usepackage{verbatim}
\makeatletter

\newdimen\errorsize \errorsize=0.2pt

% Frame with a label at top
\newcommand\LabFrame[2]{%
    \fboxrule=\FrameRule
    \fboxsep=-\errorsize
    \textcolor{FrameColor}{%
    \fbox{%
      \vbox{\nobreak
      \advance\FrameSep\errorsize
      \begingroup
        \advance\baselineskip\FrameSep
        \hrule height \baselineskip
        \nobreak
        \vskip-\baselineskip
      \endgroup
      \vskip 0.5\FrameSep
      \hbox{\hskip\FrameSep \strut
        \textcolor{TitleColor}{\textbf{#1}}}%
      \nobreak \nointerlineskip
      \vskip 1.3\FrameSep
      \hbox{\hskip\FrameSep
        {\normalcolor#2}%
        \hskip\FrameSep}%
      \vskip\FrameSep
    }}%
}}

\definecolor{FrameColor}{rgb}{0.25,0.25,1.0}
\definecolor{TitleColor}{rgb}{1.0,1.0,1.0}

\newenvironment{contlabelframe}[2][\Frame@Lab\ (cont.)]{% 
  % Optional continuation label defaults to the first label plus
  \def\Frame@Lab{#2}%
  \def\FrameCommand{\LabFrame{#2}}%
  \def\FirstFrameCommand{\LabFrame{#2}}%
  \def\MidFrameCommand{\LabFrame{#1}}%
  \def\LastFrameCommand{\LabFrame{#1}}%
  \MakeFramed{\advance\hsize-\width \FrameRestore} 
}{\endMakeFramed} 
\newcounter{definition}
\newenvironment{definition}[1]{%
  \par
  \refstepcounter{definition}%
  \begin{contlabelframe}{Definition \thedefinition:\quad #1}
 \noindent\ignorespaces}
{\end{contlabelframe}} 

%axiom
% Frame with a label at top
\newcommand\AxiomLabFrame[2]{%
    \fboxrule=\FrameRule
    \fboxsep=-\errorsize
    \textcolor{AxiomFrameColor}{%
    \fbox{%
      \vbox{\nobreak
      \advance\FrameSep\errorsize
      \begingroup
        \advance\baselineskip\FrameSep
        \hrule height \baselineskip
        \nobreak
        \vskip-\baselineskip
      \endgroup
      \vskip 0.5\FrameSep
      \hbox{\hskip\FrameSep \strut
        \textcolor{AxiomTitleColor}{\textbf{#1}}}%
      \nobreak \nointerlineskip
      \vskip 1.3\FrameSep
      \hbox{\hskip\FrameSep
        {\normalcolor#2}%
        \hskip\FrameSep}%
      \vskip\FrameSep
    }}%
}}
\definecolor{AxiomFrameColor}{rgb}{0.1,0.7,0.5}
\definecolor{AxiomTitleColor}{rgb}{1.0,1.0,1.0}

\newenvironment{acontlabelframe}[2][\Frame@Lab\ (cont.)]{% 
  % Optional continuation label defaults to the first label plus
  \def\Frame@Lab{#2}%
  \def\FrameCommand{\AxiomLabFrame{#2}}%
  \def\FirstFrameCommand{\AxiomLabFrame{#2}}%
  \def\MidFrameCommand{\AxiomLabFrame{#1}}%
  \def\LastFrameCommand{\AxiomLabFrame{#1}}%
  \MakeFramed{\advance\hsize-\width \FrameRestore} 
}{\endMakeFramed} 
\newcounter{axiom}
\newenvironment{axiom}[1]{%
  \par
  \refstepcounter{axiom}%
  \begin{acontlabelframe}{Axiom \theaxiom:\quad #1}
 \noindent\ignorespaces}
{\end{acontlabelframe}} 

% theorem
% Frame with a label at top
\newcommand\TheoremLabFrame[2]{%
    \fboxrule=\FrameRule
    \fboxsep=-\errorsize
    \textcolor{TheoremFrameColor}{%
    \fbox{%
      \vbox{\nobreak
      \advance\FrameSep\errorsize
      \begingroup
        \advance\baselineskip\FrameSep
        \hrule height \baselineskip
        \nobreak
        \vskip-\baselineskip
      \endgroup
      \vskip 0.5\FrameSep
      \hbox{\hskip\FrameSep \strut
        \textcolor{TheoremTitleColor}{\textbf{#1}}}%
      \nobreak \nointerlineskip
      \vskip 1.3\FrameSep
      \hbox{\hskip\FrameSep
        {\normalcolor#2}%
        \hskip\FrameSep}%
      \vskip\FrameSep
    }}%
}}

\definecolor{TheoremFrameColor}{rgb}{1.0,0,0}
\definecolor{TheoremTitleColor}{rgb}{1.0,1.0,1.0}

\newenvironment{tcontlabelframe}[2][\Frame@Lab\ (cont.)]{% 
  % Optional continuation label defaults to the first label plus
  \def\Frame@Lab{#2}%
  \def\FrameCommand{\TheoremLabFrame{#2}}%
  \def\FirstFrameCommand{\TheoremLabFrame{#2}}%
  \def\MidFrameCommand{\TheoremLabFrame{#1}}%
  \def\LastFrameCommand{\TheoremLabFrame{#1}}%
  \MakeFramed{\advance\hsize-\width \FrameRestore} 
}{\endMakeFramed} 
\newcounter{theorem}
\newenvironment{theorem}[1]{%
  \par
  \refstepcounter{theorem}%
  \begin{tcontlabelframe}{Theorem \thetheorem:\quad #1}
 \noindent\ignorespaces}
{\end{tcontlabelframe}} 


\makeatother
\begin{document}
%
\section*{Real and Complex Numbers}

\begin{definition}{Order}
Let $S$ be a set.  An $\textit{order}$ on $S$ is a relation, denoted by $<$, with the following two properties:
\begin{enumerate}
\item
If $x \in S$ and $y \in S$ then one and only one of the statements
$$x < y, x = y, y < x$$
is true.
\item
If $x,y,z \in S$, if $x < y$ and $y < z$, then $x < z$
\end{enumerate}
\end{definition}

\begin{definition}{Ordered Set}
An $\textit{ordered set}$ is a set $S$ in which an order is defined
\end{definition}

\begin{definition}{Upper Bound/Bounded Above}
Suppose $S$ is an ordered set, and $E \subset S$.  If there exists a $\beta \in S$ such that for every $x \in E$, we say that $E$ is $\textit{bounded above}$, and call $\beta$ an upper bound of $E$
\end{definition}

\begin{definition}{Least Upper Bound}
Suppose $S$ is an ordered set, $E \subset S$, and $E$ is bounded above.  Suppose there exists an $\alpha \in S$ with the following properties:
\begin{enumerate}
\item $\alpha$ is an upper bound of $E$
\item If $\gamma < \alpha$ then $\gamma$ is not an upper bound of $E$
\end{enumerate}
Then $\alpha$ is called the $\textit{least uppoer bound of E}$ or the $\textit{supremum of E}$ and we write:
$$\alpha = \textnormal{sup E}$$
\end{definition}

\begin{definition}{Greatest Lower Bound}
The $\textit{greatest lower bound of E}$ or the $\textit{infimum of E}$ is defined the same way.  The statement:
$$\alpha = \textnormal{inf E}$$
means that $\alpha$ is a lower bound of $E$ and that no $\beta$ with $\beta > \alpha$ is a lower bound of $E$.
\end{definition}

\begin{definition}{Least Upper Bound Property}
An ordered set is said to have a $\textit{least-upper-bound-property}$ if, for $E \subset S$, $E$ is not empty, and $E$ is bounded above, implies sup $E$ exists in $S$
\end{definition}

\begin{theorem}{1.11}
Suppose $S$ is an ordered set with the least-upper-bound property, $B \subset S$, $B$ is not empty, and $B$ is bounded below.  Let $L$ be the set of all lower bound of $B$.  Then:
$$\alpha = \textnormal{sup L}$$
exists in $S$, and $\alpha = \textnormal{inf B}$.  In particular , inf $B$ exists in $S$.
\end{theorem}

\section*{Basic Topology}

\section*{Sequences}

\section*{Series}
\begin{axiom}{Absolutely Convergent}
$\Sigma (a_n)$ is absolutely convergent if and only if $\Sigma |a_n|$ is convergent
\end{axiom}

\begin{definition}{Absolutely Convergent}
$\Sigma (a_n)$ is absolutely convergent if and only if $\Sigma |a_n|$ is convergent
\end{definition}

\begin{definition}{Conditionally Convergent}
$\Sigma (a_n)$ is conditionally convergent if and only if $\Sigma a_n$ is convergent but $\Sigma |a_n|$ is not
\end{definition}

\end{document}