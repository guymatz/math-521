\documentclass{article}
\usepackage{color,amsmath,amssymb}
\usepackage{framed}
\usepackage{verbatim}
\makeatletter

\newdimen\errorsize \errorsize=0.2pt


\newcommand{\benu}{\begin{enumerate}}
\newcommand{\eenu}{\end{enumerate}}
\newcommand{\im}{\item}
\newcommand{\mbZ}{\mathbb{Z}}
\newcommand{\mbR}{\mathbb{R}}
\newcommand{\mbN}{\mathbb{N}}
\newcommand{\mbQ}{\mathbb{Q}}
\newcommand{\mcP}{\mathcal{P}}
\newcommand{\mc}[1]{\mathcal{#1}}
\newcommand{\hra}{\hookrightarrow}
\newcommand{\tra}{\twoheadrightarrow}
\newcommand{\lra}{\leftrightarrow}
\newcommand{\ep}{\epsilon}

% Frame with a label at top
\newcommand\LabFrame[2]{%
    \fboxrule=\FrameRule
    \fboxsep=-\errorsize
    \textcolor{FrameColor}{%
    \fbox{%
      \vbox{\nobreak
      \advance\FrameSep\errorsize
      \begingroup
        \advance\baselineskip\FrameSep
        \hrule height \baselineskip
        \nobreak
        \vskip-\baselineskip
      \endgroup
      \vskip 0.5\FrameSep
      \hbox{\hskip\FrameSep \strut
        \textcolor{TitleColor}{\textbf{#1}}}%
      \nobreak \nointerlineskip
      \vskip 1.3\FrameSep
      \hbox{\hskip\FrameSep
        {\normalcolor#2}%
        \hskip\FrameSep}%
      \vskip\FrameSep
    }}%
}}

% color
\definecolor{FrameColor}{rgb}{0.25,0.25,1.0}
\definecolor{TitleColor}{rgb}{1.0,1.0,1.0}
% black and white
\definecolor{FrameColor}{rgb}{0.9,0.9,0.9}
\definecolor{TitleColor}{rgb}{0.0,0.0,0.0}

\newenvironment{contlabelframe}[2][\Frame@Lab\ (cont.)]{% 
  % Optional continuation label defaults to the first label plus
  \def\Frame@Lab{#2}%
  \def\FrameCommand{\LabFrame{#2}}%
  \def\FirstFrameCommand{\LabFrame{#2}}%
  \def\MidFrameCommand{\LabFrame{#1}}%
  \def\LastFrameCommand{\LabFrame{#1}}%
  \MakeFramed{\advance\hsize-\width \FrameRestore} 
}{\endMakeFramed} 
\newcounter{definition}
\newenvironment{definition}[1]{%
  \par
  \refstepcounter{definition}%
  \begin{contlabelframe}{Definition \thedefinition:\quad #1}
 \noindent\ignorespaces}
{\end{contlabelframe}} 

%axiom
% Frame with a label at top
\newcommand\AxiomLabFrame[2]{%
    \fboxrule=\FrameRule
    \fboxsep=-\errorsize
    \textcolor{AxiomFrameColor}{%
    \fbox{%
      \vbox{\nobreak
      \advance\FrameSep\errorsize
      \begingroup
        \advance\baselineskip\FrameSep
        \hrule height \baselineskip
        \nobreak
        \vskip-\baselineskip
      \endgroup
      \vskip 0.5\FrameSep
      \hbox{\hskip\FrameSep \strut
        \textcolor{AxiomTitleColor}{\textbf{#1}}}%
      \nobreak \nointerlineskip
      \vskip 1.3\FrameSep
      \hbox{\hskip\FrameSep
        {\normalcolor#2}%
        \hskip\FrameSep}%
      \vskip\FrameSep
    }}%
}}
% Color
\definecolor{AxiomFrameColor}{rgb}{0.1,0.7,0.5}
\definecolor{AxiomTitleColor}{rgb}{1.0,1.0,1.0}
% black and white
\definecolor{AxiomFrameColor}{rgb}{0.8,0.8,0.8}
\definecolor{AxiomTitleColor}{rgb}{0.0,0.0,0.0}

\newenvironment{acontlabelframe}[2][\Frame@Lab\ (cont.)]{% 
  % Optional continuation label defaults to the first label plus
  \def\Frame@Lab{#2}%
  \def\FrameCommand{\AxiomLabFrame{#2}}%
  \def\FirstFrameCommand{\AxiomLabFrame{#2}}%
  \def\MidFrameCommand{\AxiomLabFrame{#1}}%
  \def\LastFrameCommand{\AxiomLabFrame{#1}}%
  \MakeFramed{\advance\hsize-\width \FrameRestore} 
}{\endMakeFramed} 
\newcounter{axiom}
\newenvironment{axiom}[1]{%
  \par
  \refstepcounter{axiom}%
  \begin{acontlabelframe}{Axiom \theaxiom:\quad #1}
 \noindent\ignorespaces}
{\end{acontlabelframe}} 

% theorem
% Frame with a label at top
\newcommand\TheoremLabFrame[2]{%
    \fboxrule=\FrameRule
    \fboxsep=-\errorsize
    \textcolor{TheoremFrameColor}{%
    \fbox{%
      \vbox{\nobreak
      \advance\FrameSep\errorsize
      \begingroup
        \advance\baselineskip\FrameSep
        \hrule height \baselineskip
        \nobreak
        \vskip-\baselineskip
      \endgroup
      \vskip 0.5\FrameSep
      \hbox{\hskip\FrameSep \strut
        \textcolor{TheoremTitleColor}{\textbf{#1}}}%
      \nobreak \nointerlineskip
      \vskip 1.3\FrameSep
      \hbox{\hskip\FrameSep
        {\normalcolor#2}%
        \hskip\FrameSep}%
      \vskip\FrameSep
    }}%
}}

% color
\definecolor{TheoremFrameColor}{rgb}{1.0,0,0}
\definecolor{TheoremTitleColor}{rgb}{1.0,1.0,1.0}
% black and white
\definecolor{TheoremFrameColor}{rgb}{0.7,0.7,0.7}
\definecolor{TheoremTitleColor}{rgb}{0.0,0.0,0.0}

\newenvironment{tcontlabelframe}[2][\Frame@Lab\ (cont.)]{% 
  % Optional continuation label defaults to the first label plus
  \def\Frame@Lab{#2}%
  \def\FrameCommand{\TheoremLabFrame{#2}}%
  \def\FirstFrameCommand{\TheoremLabFrame{#2}}%
  \def\MidFrameCommand{\TheoremLabFrame{#1}}%
  \def\LastFrameCommand{\TheoremLabFrame{#1}}%
  \MakeFramed{\advance\hsize-\width \FrameRestore} 
}{\endMakeFramed} 
\newcounter{theorem}
\newenvironment{theorem}[1]{%
  \par
  \refstepcounter{theorem}%
  \begin{tcontlabelframe}{Theorem \thetheorem:\quad #1}
 \noindent\ignorespaces}
{\end{tcontlabelframe}} 


\makeatother
\begin{document}
%
\section*{Real Numbers}

\begin{definition}{Dedekind Cut}
A cut in $\mbQ$ is a pair of subsets $A, B$ of $\mbQ$ such that
\benu
\im $A \cup B = \mbQ, A \neq \emptyset, B \neq \emptyset, A \cap B \neq \emptyset$
\im If $a \in A$ and $b \in B$ then $a < b$
\im $A$ contains no largest element 
\eenu
\end{definition}

\begin{definition}{Real Number}
A \textbf{real number} is a cut in $\mbQ$
\end{definition}


\begin{definition}{less than or equal}
The cut $x = A|B$ is \textbf{less than or equal} to the cut $y = C|D$ if $A \subset C$
\end{definition}

\begin{definition}{Bounded Above}
$M \in \mbR$ is as \textbf{upper bound} for a set $S \subset \mbR$ is each $s \in S$ satifies
$$ s \leq M$$
\end{definition}

\begin{definition}{Least Upper Bound}
Un upper bound for the $S$ that is less thanall other upper bounds for $S$ is a \textbf{least upper bound} for $S$
\end{definition}


\begin{definition}{Upper Bound/Bounded Above}
	Suppose $S$ is an ordered set, and $E \subset S$.  If there exists a $\beta \in S$ such that for every $x \in E$, we say that $E$ is $\textit{bounded above}$, and call $\beta$ an upper bound of $E$
\end{definition}

\begin{definition}{Least Upper Bound}
	Suppose $S$ is an ordered set, $E \subset S$, and $E$ is bounded above.  Suppose there exists an $\alpha \in S$ with the following properties:
	\begin{enumerate}
		\item $\alpha$ is an upper bound of $E$
		\item If $\gamma < \alpha$ then $\gamma$ is not an upper bound of $E$
	\end{enumerate}
	Then $\alpha$ is called the $\textit{least uppoer bound of E}$ or the $\textit{supremum of E}$ and we write:
	$$\alpha = \textnormal{sup E}$$
\end{definition}

\begin{definition}{Greatest Lower Bound}
	The $\textit{greatest lower bound of E}$ or the $\textit{infimum of E}$ is defined the same way.  The statement:
	$$\alpha = \textnormal{inf E}$$
	means that $\alpha$ is a lower bound of $E$ and that no $\beta$ with $\beta > \alpha$ is a lower bound of $E$.
\end{definition}

\begin{definition}{Least Upper Bound Property}
	An ordered set is said to have a $\textit{least-upper-bound-property}$ if, for $E \subset S$, $E$ is not empty, and $E$ is bounded above, implies sup $E$ exists in $S$
\end{definition}


\begin{definition}{Ordered Set}
	An $\textit{ordered set}$ is a set $S$ in which an order is defined
\end{definition}

\begin{theorem}{1.11}
	Suppose $S$ is an ordered set with the least-upper-bound property, $B \subset S$, $B$ is not empty, and $B$ is bounded below.  Let $L$ be the set of all lower bound of $B$.  Then:
	$$\alpha = \textnormal{sup L}$$
	exists in $S$, and $\alpha = \textnormal{inf B}$.  In particular , inf $B$ exists in $S$.
\end{theorem}


\begin{definition}{Trichotomy}
Either $x < y, y < x$ or $x = y$, but only one of these is true
\end{definition}

\begin{definition}{Cauchy Sequence}
The sequence $(a_n)$ \textbf{converges} to the limit $b \in \mbR$ as $n \rightarrow \infty$ provided that for each $\ep > 0$ there exists $N \in \mbN$ such that  for all $n \geq N$,
$$|a_n - b_n| < \ep$$ 
\end{definition}


\begin{definition}{Cauchy Convergence Criterion for Sequences}
	A sequence $|a_n - a_m| < \ep$ in $\mbR$ converges if and only if
	$$\forall \ep > 0 \exists N \in \mbN \text{ such that } n,m \geq N \Rightarrow |a_n - a_m| < \ep$$ 
\end{definition}


\begin{definition}{(a,b)}
	$(a,b) = \{x \in \mbR : a < x < b\}$
\end{definition}

\begin{definition}{[a,b]}
$[a,b] = \{x \in \mbR : a \leq x \leq b\}$
\end{definition}


\begin{definition}{Archimedean Property}
	For each $x \in \mbR$ there is an integer $n$ that is greater than $x$
\end{definition}


\begin{definition}{Dot Product  (Scalar Product)}
	The \textbf{dot product} of the vector $x = (x_1, x_2, \dots, x_n)$ and $y = (y_1, y_2, \dots, y_n)$ is
	$$\langle x, y \rangle = x_1y_1 + \dots x_my_m$$
\end{definition}


\begin{definition}{Inner Product}
	An \textbf{inner product} on a vector space, V, is an operation < , > on pairs of vectors that satisfies the same conditions that the dot product in Euclidean space does
\end{definition}


\begin{definition}{Triangle Inequality for Vectors}
	For all $x,y \in \mbR^m$
	$$|x + y| \leq |x| + |y|$$
\end{definition}

\begin{definition}{Range or Image}
	The \textbf{range} of \textbf{image} of $f$ is the subset of the target,
	$$\{b \in B: \text{there exists at least one element} a \in A \text{ with } f(a) = b\}$$
\end{definition}

\begin{definition}{Injection / one-to-one}
	A mapping $f: A \rightarrow B$ is an \textbf{injection} or is \textbf{one-to-one} if for each pair of distinct elements $a, a' \in A$ the elements $f(a), f(a')$ are distinct in B.  That is,
	$$a \neq a' \Rightarrow f(a) \neq f(a')$$
\end{definition}


\begin{definition}{Surjection / onto}
	The mapping $f$ is a \textbf{surjection} (or is \textbf{onto}) if for each $b \in B$ there is at least one $a \in A$ such that $f(a) = b$.  That is, the range of $f$ is $B$
\end{definition}


\begin{definition}{Identity Map}
	The bijection that takes each $a \in A$ and sends it to itself
	$$id(a) = a$$
\end{definition}


\begin{definition}{Composite}
If $f:A \to B$ and $g:B \to C$ then the \textbf{composite} $g \circ f:A \to C$ is the function that sends $a \in A$ to $g(f(a)) \in C$
\end{definition}

\begin{definition}{Equivalence}
	The relation $\sim$ is an equivalence relation
\benu
	\im $A \sim A$
	\im If $A \sim B$, then $B \sim A$
	\im If $A \sim B$, and $B \sim C$, then $A \sim C$
\eenu
\end{definition}

\begin{definition}{Finite}
	A set $S$ is \textbf{finite} if it is empty or for some $n \in \mbN$, $S \sim \{1, \dots, n\}$
\end{definition}

\begin{definition}{Infinite}
	A set $S$ is \textbf{infinite} if it is not finite
\end{definition}

\begin{definition}{Denumerable}
	A set $S$ is \textbf{denumerable} if $S \sim \mbN$
\end{definition}

\begin{definition}{Countable}
	A set $S$ is \textbf{countable} if it is finite or denumerable
\end{definition}

\begin{definition}{Uncountable}
	A set $S$ is \textbf{uncountable} if it is not countable
\end{definition}


\begin{definition}{Order}
Let $S$ be a set.  An $\textit{order}$ on $S$ is a relation, denoted by $<$, with the following two properties:
\begin{enumerate}
\item
If $x \in S$ and $y \in S$ then one and only one of the statements
$$x < y, x = y, y < x$$
is true.
\item
If $x,y,z \in S$, if $x < y$ and $y < z$, then $x < z$
\end{enumerate}
\end{definition}

\newpage

\section*{A Taste of Topology}

\begin{definition}{Metric Space}
A \textbf{metric space} is a pair $(M, d)$, where $M$ is a set and $d$ is a distance function on pairs of $M$, so that:
$$d: M^2 \to \mbR$$
\benu
\im Positive Definiteness: $d(x,y) \geq 0$ and $d(x,y) = 0$ iff $x = y$
\im Symmetry: $d(x,y) = d(y,x)$
\im Triangle Inequality: $d(x,z) \leq d(x,y) + d(y,z)$
\eenu
\end{definition}

\begin{definition}{Sequence}
A \textbf{sequence} in $M$ is a list of elements of $M$ indexed by $\mbN$, $(a_i)_{i \in \mbN} \text{ or } (a_i)$
Formally, a \textbf{sequence} is a function $f: \mbN \to M$, the metric space
\end{definition}


\begin{definition}{Convergence}
	The sequence $(p_n)_{n \in \mbN}$ \textbf{converges to the limit} $p$ in $M$ if:
	$$\forall \ep > 0  \, \exists N \in \mbN \text{ such that}$$
	$$n \in \mbN \text{ and } n \geq N \implies d(p_n,p) < \ep$$
\end{definition}


\begin{definition}{Convergence}
	The sequence $(a_i)_{i \in \mbN}$ in $M$ \textbf{ converges } to $b \in M$ if:
	$$\forall\ep > 0  \, \exists N \in \mbN \text{ such that}$$
	$$\forall j > N  \, d(a_j,b) < \ep$$
\end{definition}

\begin{theorem}{}
	Every subsequence of a convergent sequence converges and it converges to the same limit as does the mother sequence
\end{theorem}

\begin{definition}{Subsequence}
	If $(p_n)_{n \in \mbN}$ and $(q_n)_{n \in \mbN}$ are sequences and if there is a sequence $1 \leq n_1 < n_2 < n_3 < \dots$ of integers such that for each $k \in \mbN, q _k = p_{n_k}$ then $(q_k)$ is a \textbf{subsequence} of $(p_n)$
\end{definition}


\begin{definition}{Continuous I}
	A function $f:M \to N$ is \textbf{continuous} if whenever $(a_i)$ is a sequence in $M$ and
$$(a_i) \to b \text{ in } M,$$
$$f((a_i)) \to f(b) \text{ in } N$$
\end{definition}

\begin{theorem}{}
	$f : M \to N$ is continuous iff it sends each convergent sequence in $M$ to a convergent sequence in $N$, limits being sent to limits.
\end{theorem}

\begin{theorem}{}
	The composite of continuous functions is continuous
\end{theorem}

\begin{definition}{Continuous II }
	A function $f:M \to N$ is \textbf{continuous} if it satisfies the $\boldsymbol{\epsilon, \delta}$ \textbf{condition}:
	$\forall \epsilon > 0 \text{ and } \forall p \in M \, \exists \delta > 0 \text{ such that }$:
	$$q \in M \text{and } d(p,q) < \delta \implies d(fp,fq) < \epsilon$$
\end{definition}

\begin{definition}{Continuous III }
	A function $f:M \to N$ is \textbf{continuous} if it satisfies the $\boldsymbol{\epsilon, \delta}$ \textbf{condition}:
	$\forall \epsilon > 0 \text{ and } \forall p \in M \, \exists \delta > 0 \text{ such that }$:
	$$q \in M \text{and } d(p,q) < \delta \implies d(fp,fq) < \epsilon$$
\end{definition}


\begin{definition}{Continuous IV - Continu-ish}
	A function $f:M \to N$ is \textbf{continu-ish} if
	$\forall x \in M \forall \epsilon > 0 \exists \delta > 0 \forall y \in M$,\\
	$$d_M(x,y) < \delta \implies d_N(f(x), f(y)) < \epsilon$$
\end{definition}


\begin{definition}{Homeomorphism}
	If $f: M \to N$ is a bijection and $f$ is continuous and the inverse bijection $f^{-1}:N \to M$ is also continuous then $f$ is a \textbf{homeomorphism}
	($M \cong N$)
\end{definition}


\begin{definition}{Limit Point}
	For $S \subseteq M, p \in M, p$ is a \textbf{Limit point} of $S$ if there is a sequence $(a_i)$ so that $(a_i) \to p$ and $(a_i) \in S$
	$$lim^M(S) = \{q \in M|q \text{ is a limit of }S\}$$
\end{definition}

\begin{definition}{Closedness}
	\begin{center}
	$S \subseteq M$ is \textbf{closed} in $M$ if it contains all of its limit points in $M$.
	\\ OR \\
	$S \subseteq M$ is \textbf{closed} in $M$ if $lim^M(S) \subseteq S$
	\\ OR \\
	$S \subseteq M$ is \textbf{closed} in $M$ if $lim^M(S) = S$
	\\ OR \\
	$V \subseteq M$ is \textbf{closed} in $M$ if $lim(V) \subseteq V$, where $x \in lim(V)$ if there is a seq($a_i$) in $V$ so $(a_i) \to x$ 
    \\ OR \\
    If $(a_i) \to b$ where each $a_i \in f^{\text{pre}}(K)$, then $b \in f^{\text{pre}}(K)$
    
	\end{center}
\end{definition}


\begin{definition}{Openness}
	A set $U \subseteq M$ is \textbf{open in M} if for every point $p \in U$ there is an $r \in \mbR$, $r>0$, so that for every $q \in M$,
	if $d(p,q) < r$, then $q < U$
	\begin{center}
	OR
	\\
	$\forall p \in S \, \exists r > 0 (B_r^M(p) \subseteq S)$
	\end{center}
\end{definition}

\begin{definition}{Topology}
The \textbf{topology} of $M$ is the collection $\mc{T}$ of all open subsets of $M$.
\end{definition}

\begin{theorem}{Complements of Closed Sets are Open}
If $V$ is closed in $M$, then $M \setminus V$ is open in $M$
\end{theorem}

\begin{theorem}{$\mathbf{\mc{T}}$ satisfies three properties}
\benu
\item Every \textbf{union} of open sets is an open set
\item The \textbf{intersection} of finitely many members of $\mc{T}$ is a member of $\mc{T}$ 
\item $\emptyset$ and $M \subseteq \mc{T}$
\eenu
\end{theorem}

\begin{theorem}{Corollary to $\mathbf{\mc{T}}$ satisfies three properties}
\benu
\item The intersection of any number of closed sets is a closed set
\item The finite union of closed sets is a closed set
\item $\emptyset$ and $M$ are closed sets
\eenu
\end{theorem}

\begin{theorem}{Intersectin/Union of Closed Sets}
	Any intersection of closed sets is closed and and union of finitely many closed sets is closed
\end{theorem}

\begin{definition}{$f^{\text{pre}}$}
	For $f : M \to N, Y \subseteq N$,
	$$f^{\text{pre}}(Y) = \{x \in M | f(x) \in Y\}$$
\end{definition}

\begin{theorem}{Continuous == Open/Closed}
	The following are equivalent:
	\benu
	\item $f$ is continuous
	\item $f$ is continu-ish
	\item For every closed set $K \subseteq M$, $f^{\text{pre}}(K)$ is closed in $M$
	\item For every open set $U \subseteq N$, $f^{\text{pre}}(U)$ is open in $M$
	\eenu
\end{theorem}

\begin{theorem}{Cor. of Continuiuty/Open/Closed Equivalances}
	If $f : M \to N$ is a homemoorphism, then for every $X \subseteq M$
	$$X \text{ is open in } M \Longleftrightarrow \{f(p) | p \in X\} \text{ is open in }N$$
\end{theorem}
\begin{definition}{r-neighborhood of p}
  The \textbf{r-neighborhood} of p - notated $M_rp = \{q \in M : d(p,q) < r \}$
\end{definition}

\begin{definition}{limit set of S}
  The \textbf{limit set} of S - notated $\text{lim } S = \{p \in M : p \text{ is a limit of } S \}$
\end{definition}

\begin{theorem}{lim $S$ is a closed set}
\end{theorem}

\begin{theorem}{$M_rp$ is an open set}
\end{theorem}

\begin{theorem}{lim(S) is the smallest closed set containing S}
	If $S \subseteq X$ is closed, then $\text{lim}(S) \subseteq X$
\end{theorem}

\begin{theorem}{The intersection of all closed sets equals lim(S)}
	${\bigcap}_{X \text{ closed }, S \leq X} = \text{lim}(S)$
\end{theorem}

\begin{definition}{$\bar{S}$}
	lim(S) = cl(S) = $\bar{S}$
\end{definition}

\begin{theorem}{Limits sets of limits sets equal limits sets}
	lim(lim($S$)) = lim(S)\\
	cl(cl(S)) = cl(S)\\
	$\bar{\bar{S}} = \bar{S}$
\end{theorem}

\begin{theorem}{The following are equivalent}
  \benu
\item $p$ is a limit of $S$
\item $\forall r > 0, M_r(p) \cap S \neq \emptyset$
  \eenu
\end{theorem}

\begin{theorem}{If $S \subseteq N \subseteq M$ and $S$ is open in $M$}
	$$S \text{ is open in } N$$
	\\
	I.e., Open in the big space implies open in the small space
\end{theorem}

\begin{theorem}{If $S \subseteq N \subseteq M$ and $S$ is closed in $M$}
	$$S \text{ is closed in } N$$
	\\
	I.e., Closed in the big space implies closed in the small space
\end{theorem}

\begin{theorem}{Let $S \subseteq N \subseteq M$}
	\begin{center}
	$S$ is open in $N$\\
	iff\\
	there is some $V \subseteq M$ such that $V$ is open in $M$ and $V \cap N = S$
\end{center}
\end{theorem}

\begin{theorem}{Let $K \subseteq N \subseteq M$}
	\begin{center}
		$K$ is closed in $N$\\
		iff\\
		there is a closed set $L \subseteq M$ such that $L$ is closed in $M$ and $L \cap N = K$
	\end{center}
\end{theorem}

\begin{theorem}{A sequence in a Cartesian product of $m$ metric spaces converges . . .}
	. . . with respect to the sum metric iff it converges with respect to the max metric  iff each component sequence converges
\end{theorem}

\begin{theorem}{Let $f: X \to M$ and $g: Y \to N$ be functions . . .}
	. . .  $f \times g: X \times Y \to M \times N$ is continuous iff $f$ is continuous and $g$ is continuous 
\end{theorem}

\begin{theorem}{In $\mbR^n$, a sequence converges iff. . .}
	. . .  each of its components of its coordinate sequences converge
\end{theorem}

\begin{theorem}{The function $d:M \times M \to \mbR$  . . .}
	. . .  is continuous for any metric space $(M,d)$
\end{theorem}

\begin{theorem}{A sequence of vectors $(v_n) \text{ in } \mbR$ converges in $\mbR$. . .}
	. . . iff each of its component sequences $(v_{in})$ converges, $i \leq i \leq m$.  The limit of the vector sequence is the vector
	$$v = \lim_{n\to\infty} v_n = \Big(\lim_{n\to\infty} v_{1n}, \lim_{n\to\infty} v_{2n}, \dots , \lim_{n\to\infty} v_{mn}\Big)$$
\end{theorem}

\begin{theorem}{Every closed subset of a complete metric space . . .}
	. . . is a complete metric space
\end{theorem}

\begin{definition}{Complete}
	The metric space $M$ is complete if every Cauchy sequence in $M$ converges in $M$
	$$S \subset M_rp$$
\end{definition}


\begin{theorem}{If $M$ is complete and $N$ is complete . . .}
    . . .  then $M \times N$ is complete
\end{theorem}


\begin{theorem}{If $N \subseteq M$, $M$ is complete and $N$ is closed in $N$. . .}
	. . .  then $N$ is complete
\end{theorem}


\begin{definition}{Bounded}
	A subset $S$ of a metric space $M$ is \textbf{bounded} if for some $p \in M$ and some $r > 0$,
	$$S \subset M_rp$$
\end{definition}

\begin{definition}{Bounded Function}
	A function from $M$ to another metric space is a \textbf{bounded function} if its range is bounded.  That is, there exist $q \in N$ and $r > 0$ such that
	$$f M \subset N_rq$$
\end{definition}

\begin{definition}{Compact}
	A subset $A$ of a metric space $M$ is \textbf{compact} if every sequence $(a_n)$ in $A$ has a subsequence $(a_{n_k})$ that converges to a limit in $A$.
\end{definition}

\begin{theorem}{}
	Finite sets are compact
\end{theorem}

\begin{theorem}{A closed interval .  .}
	. . . is compact
\end{theorem}

\begin{theorem}{Every compact set  . . .}
	. . .  is closed and bounded
\end{theorem}


\begin{theorem}{If $A \subseteq M$ is compact . . .}
	. . .  Then $A$ is closed in $M$ and bounded
\end{theorem}

\begin{theorem}{If $A$ is compact and $B$ is compact . . .}
	. . .  Then $A \times B$ is compact
\end{theorem}

\begin{theorem}{If $A_1, A_2, \dots A_k$ are compact. . .}
	. . .  Then $A_1 \times A_2 \times \dots \times A_k$ is compact
\end{theorem}

\begin{theorem}{If $A_1, A_2, \dots A_k$ are compact. . .}
	. . .  Then $A_1 \times A_2 \times \dots \times A_k$ is compact
\end{theorem}


\begin{definition}{A Closed Box in $\mbR^m$}
    A closed box in $\mbR$ is a set of the form $[a_1,b_1] \times [a_2,b_2] \dots \times [a_m,b_m]$
\end{definition}


\begin{theorem}{Every closed box in $\mbR^m$. . .}
	. . .  is compact
\end{theorem}


\begin{theorem}{If $M$ is compact and $A \subseteq M$ is closed . . .}
	. . .  then $A$ is compact
\end{theorem}

\begin{theorem}{Every bounded sequence in $\mbR^m$ . . .}
	. . .  has a convergent subsequence
\end{theorem}

\begin{theorem}{Every closed and bounded set of $\mbR^m$ . . .}
	. . .  is compact
\end{theorem}

\newpage
\section*{Homework}

\newpage

\end{document}
