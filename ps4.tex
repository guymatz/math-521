\documentclass[12pt]{amsart}
\setlength{\parskip}{.1in}
\setlength{\parindent}{0cm}
%myalterations
\usepackage{amssymb}
\usepackage{amsmath}
\usepackage[usenames,dvipsnames,svgnames,table]{xcolor}
\usepackage[colorlinks=true,urlcolor=blue,pdfborder={0 0 .5}pdfnewwindow=true]{hyperref}
\usepackage{enumitem}
%\usepackage{amsthm}
\usepackage{graphicx}
\usepackage{verbatim}
\usepackage{tabularx}
%\usepackage{arydshln,leftidx,mathtools}
\usepackage{bm}
\usepackage{tikz}
\usepackage{tikz-cd}
\usepackage{hyperref}
\usepackage{bm}

%\setlength{\dashlinedash}{.4pt}
%\setlength{\dashlinegap}{.8pt}
%\usepackage{amsthm}
\usepackage{verbatim}
%\usepackage{commath}
%My commands
%environment abbreviations
\newcommand{\benu}{\begin{enumerate}}
\newcommand{\eenu}{\end{enumerate}}
\newcommand{\bed}{\begin{description}}
\newcommand{\ed}{\end{description}}
\theoremstyle{definition}
\newtheorem{theorem}{Theorem}
\newtheorem{notation}[theorem]{Notation}
\newtheorem{exercise}[theorem]{Exercise}
\newcommand{\bex}{\begin{exercise}}
\newcommand{\ex}{\end{exercise}}

\newcommand{\pru}{{ \bfseries \textcolor{red}{Proof:} }}

%symbol definitions
\newcommand{\un}[1]{\underline{#1}}
\newcommand{\mbZ}{\mathbb{Z}}
\newcommand{\mbR}{\mathbb{R}}
\newcommand{\mbN}{\mathbb{N}}
\newcommand{\mbQ}{\mathbb{Q}}
\newcommand{\mbC}{\mathbb{C}}
\newcommand{\mbF}{\mathbb{F}}
\newcommand{\mcS}{\mathcal{S}}
\newcommand{\mcP}{\mathcal{P}}
\newcommand{\hra}{\hookrightarrow}
\newcommand{\tra}{\twoheadrightarrow}
\newcommand{\lra}{\leftrightarrow}
\newcommand{\ep}{\epsilon}
\newcommand{\Ra}{\Rightarrow}
\newcommand{\mb}[1]{\mathbb{#1}}
\newcommand{\mc}[1]{\mathcal{#1}}
\newcommand{\bfs}[1]{{\bfseries #1}}
\newcommand{\bs}[1]{\boldsymbol{#1}}
%Operator definitions
\DeclareMathOperator{\Irr}{Irr}
\DeclareMathOperator{\triv}{triv}
\DeclareMathOperator{\cyc}{cyc}
\DeclareMathOperator{\lcm}{lcm}
\DeclareMathOperator{\expo}{x}
\DeclareMathOperator{\ord}{o}
\DeclareMathOperator{\imm}{im}
\DeclareMathOperator{\sgn}{sgn}
\DeclareMathOperator{\Sym}{Sym}
\DeclareMathOperator{\alt}{alt}
\DeclareMathOperator{\irr}{irr}
\DeclareMathOperator{\eqt}{Equiv}
\DeclareMathOperator{\pat}{Part}
%\DeclareMathOperator{\sgn}{sgn}
%\DeclareMathOperator{\Aut}{Aut}
\DeclareMathOperator{\Gl}{Gl}
\DeclareMathOperator{\M}{M}
\DeclareMathOperator{\Id}{Id}
\DeclareMathOperator{\fixx}{Fix}
\DeclareMathOperator{\suppp}{Supp}
\DeclareMathOperator{\gl}{Gl}
\DeclareMathOperator{\id}{Id}
\DeclareMathOperator{\Aut}{Aut}
\DeclareMathOperator{\Inn}{Inn}
\DeclareMathOperator{\orb}{orb}
\DeclareMathOperator{\ii}{I}
\DeclareMathOperator{\im}{im}
\DeclareMathOperator{\Fix}{Fix}
\DeclareMathOperator{\Co}{Co}
\DeclareMathOperator{\md}{md}
\DeclareMathOperator{\qt}{qt}
\DeclareMathOperator{\ExtendedGCD}{ExtendedGCD}
\DeclareMathOperator{\Mod}{Mod}
\DeclareMathOperator{\GCD}{GCD}
\newcommand{\nms}{\negmedspace}
\newcommand{\nts}{\negthinspace}

\newcommand{\itep}{\item {\bfseries Problem}\ }
\newcommand{\gen}[1]{\langle \nts#1 \nts\rangle}
\newcommand{\quot}[2]{#1/ #2}
\newcommand{\order}[1]{\left|<\nts #1 \nts s>\right|}

%These next two commands are for making answers. 
\newcommand{\beans}{\begin{description} \item[{ \bfseries \textcolor{red}{Answer}}]\ }
\newcommand{\eans }{\end{description}}
%\newcommand{\begin{comment}ex}{{ \bfseries \textcolor{red}{Answer}}}

%To turn the answer into problem sets use replace to replace \begin{comment} with \begin{comment} and \\end{comment}  by \end{comment}.
\newcommand{\lieb}[3][{{}}]{\frac{d^#1 #2}{d\,#3^#1}}

\title{\textbf{Math 521 - Problem Set 4}}
\author{Guy Matz}
\date{\today}

\begin{document} 

\maketitle
\newpage % Q1

\begin{enumerate}[series=p]
\itep  If every closed and bounded subset of a metric space $M$ is compact, does it follow that $M$ is complete? (Proof or counterexample.)
\\\\
Let $S$ be a Cauchy sequence, and let $\bar{S}$ be its closure.  Since $S$ is bounded, $\bar{S}$ is as well, and $\bar{S}$ is closed.  By hypothesis, $\bar{S}$ is compact, hence it is complete..  Therefore, since $S \subseteq \bar{S}$ is a Cauchy sequence, it has a limit in $\bar{S} \subseteq M$.


\newpage

\itep Show that the 3 metrics we discussed on the product space $X \times Y$ are in fact metrics

\underline{Metric Space}
	A \textbf{metric space} is a pair $(M, d)$, where $M$ is a set and $d$ is a distance function on pairs of $M$, so that:
	$$d: M^2 \to \mbR$$
\benu
	\item Euclidean (Madison) Metric
	\benu	
	\item Positive Definiteness:
		\benu
		\item $d(x,y) \geq 0$\\
			$d_X(x, x')^2 \geq 0$, and $d_Y(y, y')^2 \geq 0$, so\\ $d_E = \sqrt{d_X(x, x')^2 + d_Y(y, y')^2} \geq 0$
		\item $d(x,y) = 0$ iff $x = y$\\
			$\Rightarrow$ Assuming $d(x,y) = 0$, WTS $x=y$\\
			$\sqrt{d_X(x, x')^2 + d_Y(y, y')^2} = 0$\\
			$d_X(x, x')^2 + d_Y(y, y')^2 = 0$\\
			So by 1.a.i.A above, $d_X(x, x')^2 = d_Y(y, y')^2 = 0$,
			so $x = y$\\
			$\Leftarrow$ Assuming $x = y$, WTS $d(x,y) = 0$\\
			$\sqrt{d_X(x, x')^2 + d_Y(y, y')^2}$\\
			$ = \sqrt{0 + 0}$\\
			$ =0$
		\eenu
	\item Symmetry: $d(x,y) = d(y,x)$\\
		$d_E((x, y), (x', y')) = \sqrt{d_X(x, x')^2 + d_Y(y, y')^2}$\\
		$= d_X(x, x')^2 + d_Y(y, y')^2$\\
		$= (x - x')^2 + (y - y')^2$\\
	    $= x^2 - 2xx' + x'^2 + y^2 -2yy' + y'^2$\\
		$= (x' - x)^2 + (y' - y)^2$\\
		$\sqrt{d_X(x', x)^2 + d_Y(y', y)^2}$\\
	\item Triangle Inequality: $d(x,z) \leq d(x,y) + d(y,z)$\\
		$0 \leq 2d_Y(y, y')^2 + 2\sqrt{d_X(x, x')^2 + d_Y(y, y')^2} \sqrt{d_Y(y, y')^2 + d_Z(z, z')^2} $\\
		and\\
		$d_X(x, x')^2 + d_Z(z, z')^2 \leq 
		d_X(x, x')^2 + d_Y(y, y')^2 + 
		2\sqrt{d_X(x, x')^2 + d_Y(y, y')^2} \sqrt{d_Y(y, y')^2 + d_Z(z, z')^2} 
		+  d_Y(y, y')^2 + d_Z(z, z')^2$\\
		so\\
		$\sqrt{d_X(x, x')^2 + d_Z(z, z')^2} \leq \sqrt{d_X(x, x')^2 + d_Y(y, y')^2} + \sqrt{d_Y(y, y')^2 + d_Z(z, z')^2}$\\
		
	\eenu
	\item Sum (Manhattan) Metric
	\benu	
		\item Positive Definiteness:
		\benu
		\item $d(x,y) \geq 0$\\
		$d_X(x, x') \geq 0$, and $d_Y(y, y') \geq 0$, so\\ $d_X(x, x') + d_Y(y, y') \geq 0$
		\item $d(x,y) = 0$ iff $x = y$\\
		$\Rightarrow$\\
		$d_X(x, x') + d_Y(y, y') = 0$\\
		$d_X(x, x') + d_Y(y, y') = 0$\\
		So by 1.b.i.A above,\\
		$d_X(x, x') = d_Y(y, y') = 0$\\
		$\Leftarrow$\\
		$d_X(x,x) + d_Y(y,y)$\\
		$= 0 + 0$\\
		$ = 0$
		\eenu
		\item Symmetry: $d(x,y) = d(y,x)$\\
		$d_{SUM}((x,y),(x',y')) = d_X(x, x') + d_Y(y, y')$\\
		$= d_X(x, x') + d_Y(y, y')$\\
		$= |(x - x')| + |(y - y')|$\\
		$= |(x' - x)| + |(y' - y)|$\\
		$= d_X(x', x) + d_Y(y', y)$\\
		$ = d_{SUM}((x',y'),(x,y))$
		\item Triangle Inequality: $d(x,z) \leq d(x,y) + d(y,z)$\\
		$0 \leq 2|y - y'|$\\
		$0 \leq 2d_Y(y, y')$\\
		$d_X(x, x') + d_Z(z, z') \leq d_X(x, x') + 2d_Y(y, y') + d_Z(z, z')$\\
		$d(x,z) \leq d(x, y) + d(y,z)$
		
\eenu
	\item Max Metric
	\benu	
\item Positive Definiteness:
\benu
\item $d(x,y) \geq 0$\\
$d_{MAX}(x,y) = \text{max} \{ d_X(x), d_Y(y)\}$\\
$d_X(x) \geq 0$, and $d_Y(y) \geq 0$, so\\ $d_{MAX}(x, y) \geq 0$
\item $d(x,y) = 0$ iff $x = y$\\
$\Rightarrow$ Assuming $d_{MAX}(x,y) = 0$, WTS $x=y$\\
$d_{MAX}(x,y) = \text{max}\{d_X(x, x'), d_Y(y, y')\} = 0$\\
So by 1.c.i.A above,\\
$d_X(x) = d_Y(y) = 0$\\
$\Leftarrow$ Assuming $x = y$ WTS $d_{MAX}(x,y)=0$\\
$d_{MAX} = \text{max}\{d_X(x) + d_Y(y)\}$\\
$= \text{max}\{0 + 0\}$\\
$ = 0$
\eenu
\item Symmetry: $d(x,y) = d(y,x)$\\
$d_{MAX}((x,y)) = \text{max}\{d_X(x, x'), d_Y(y, y')\}$\\
$ = \text{max}\{d_X(x', x), d_Y(y', y)\}$\\
$ = d_{MAX}((x',y'))$
\item Triangle Inequality: $d_{MAX}(x,z) \leq d_{MAX}(x,y) + d_{MAX}(y,z)$\\
$\text{max}\{d_X(x,x'), d_Z(z,z')\} \leq \text{max}\{d_X(x,x'), d_Y(y,y')\} + \text{max}\{d_Y(y,y'), d_Z(z,z')\}$\\
\benu
\item If $x > y > z$,\\
$d_X(x,x') \leq d_X(x,x') + d_Y(y,y')$, which is true\\
\item If $x > z > y$,\\
$d_X(x,x') \leq d_X(x,x') + d_Y(z,z')$, which is true\\
\item If $y > x > z$,\\
$d_X(x,x') \leq d_Y(y,y') + d_X(x,x')$, which is true\\
\item If $y > z > x$,\\
$d_Z(z,z') \leq d_Y(y,y') + d_Z(z,z')$, which is true\\
\item If $z > x > y$,\\
$d_Z(z,z') \leq d_X(x,x') + d_Z(z,z')$, which is true\\
\item If $z > y > x$,\\
$d_Z(z,z') \leq d_Y(y,y') + d_Z(z,z')$, which is true\\
\eenu
\eenu
\eenu

\newpage

\itep 38\\
Let $||$ $||$ be any norm on $\mbR^m$ and let $B = \{x \in \mbR^m : ||x|| \leq 1 \}$.  Prove that $B$ is compact.  [Hint: It suffices to show that $B$ is closed and bounded with respect to the Euclidean metric.]

\textbf{Definition: Norm}\\
A \textbf{norm} on a vector space $V$ is any function $\lVert$   $\rVert: V \to \mbR$ with the three properties of vector length: namely, if $v, w \in V$ and $\lambda \in \mbR$ then\\
$\lVert v \rVert \geq 0$ and $\lVert v \rVert = 0$ iff $v = 0$,\\
$\lVert \lambda v \rVert = |\lambda| \lVert v \rVert$,\\
$\lVert v + w \rVert \leq v \rVert + \lvert w \rVert$
\newpage


\itep 39\\
Assume that the Cartesian product of two non-empty set $A \subset M$ and $B \subset N$ is compact in $M \times N$.  Prove that $A$ and $B$ are compact.
\\\\
Given $A \times B$ is compact, we want to show $A$ and $B$ are compact.  Let $p_n=((a_n), (b_n))$ be a sequence in $A \times B$.  Since $A \times B$ is compact, we have that the subsequence $p_{n_s} \to p = (c,d)$.  Then $(a_n)$ has a convergent subsequence $(a_{n_k}) \to c \in A$ and $(b_n)$ has a convergent subsequence $(b_{n_l}) \to b \in B$, so both $A$ and $B$ are compact.
\newpage

\itep 40\\
Consider a function $f:M \to \mbR$.  Its graph is the set
$$\{(p,y) \in M \times \mbR: y = fp\}.$$
\benu
\item Prove that if $f$ is continuous then its graph is closed (as a subset of $M \times \mbR$)
\\\\
If $f$ is continuous, then $(a_i)$ in M, and $(a_i) \to b$ in $M$ implies that $f((a_i)) \to f(b)$ in $N$.  Hence its graph is closed
\\
If the graph of $f$ is compact, then evry sequence in the graph has a subsequence that converges to a limit in the graph.  Hence $f$ is continuous
\item Prove that if $f$ is continuous and $M$ is compact then its graph is compact
\\\\
We want to show that every sequence in the graph, $N$, has a sequence that converges to a limit.  Since $M$ is compact, every subsequence in $M$ converges to a limit, and since it is continuous, $(a_i) \to b$ in $M$ implies that $f((a_i)) \to f(b)$ in $N$.  Therefore every sequence in $N$ converges to a limit, and so every subsequence does as well.  Hence $N$ is compact.

\item Prove that if the graph of $f$ is compact then $f$ is continuous
\\\\
If the graph of $f$ is compact, then every sequence in the graph has a subsequence that converges to a limit in the graph.  Since a function is continuous if whenever $(a_i) \to b$ in $M$, and $f((i_i)) \to f(b)$ in $N$, it follows that $f$ is continuous
\\
\item What if the graph is merely closed?  Give an example of a discontinuous function $f: \mbR \to \mbR$ whose graph is closed
\\
$f(x) = 
\begin{cases}
tan(x)
\\0,  x = \pi/2 + n\pi: n \in \mbZ
\end{cases}
$
\eenu
\newpage

\itep 27\\
Let $(A_n)$ be a nested decreasing sequence of non-empty closed sets in the metric space $M$.
\benu
\item If $M$ is complete and diam $A_n \to 0$ as $n \to \infty$, show that $\cap A_n$ is exactly one point\\
\\
We first want to show that the intersection has at least one point.  So we want to show that a Cauchy sequence exists.  Let $a_n \in A_n$ be no-empty.  Because they are nested and diam $\to 0$, the sequence is Cauchy.  $M$ is therefore complete and its limit is $p$.  Since $A_n$ is closed, $p \in A_n$ since there is a sequence that tends to $p$.  So then $p$ is in the intersection.
\\
To show that there are at most one point in the intersection, suppose TAC $a, b \in A_n$ and $a \neq b$.  Then $d(a, b) = r > 0$.  Let $n$ be so diam($A_n) < r$.  But $a, b \in A_n,$ and $r > \text{ diam}(A_n) \geq d(a, b) = r$.  So then $r > r$. $\hfill$CONTRADICTION
\\
\item To what assertions do the sets $[n, \infty)$ provide counterexamples?
\\
Let $A_n \in M = \mbR$ be nested and decreasing, and $M$ is a complete metric space.  As well, $A_n = [n, \infty)$.  But then diam($A_n) = \infty$ for all $n$, so it is not going to 0.
\eenu
\newpage

\end{enumerate}
\end{document}
