\documentclass[12pt]{amsart}
\setlength{\parskip}{.1in}
\setlength{\parindent}{0cm}
%myalterations
\usepackage{amssymb}
\usepackage{amsmath}
\usepackage[usenames,dvipsnames,svgnames,table]{xcolor}
\usepackage[colorlinks=true,urlcolor=blue,pdfborder={0 0 .5}pdfnewwindow=true]{hyperref}
\usepackage{enumitem}
%\usepackage{amsthm}
\usepackage{graphicx}
\usepackage{verbatim}
\usepackage{tabularx}
%\usepackage{arydshln,leftidx,mathtools}
\usepackage{bm}
\usepackage{tikz}
\usepackage{tikz-cd}
\usepackage{hyperref}
\usepackage{bm}
\usepackage{wasysym}


%\setlength{\dashlinedash}{.4pt}
%\setlength{\dashlinegap}{.8pt}
%\usepackage{amsthm}
\usepackage{verbatim}
%\usepackage{commath}
%My commands
%environment abbreviations
\newcommand{\benu}{\begin{enumerate}}
\newcommand{\eenu}{\end{enumerate}}
\newcommand{\bed}{\begin{description}}
\newcommand{\ed}{\end{description}}
\theoremstyle{definition}
\newtheorem{theorem}{Theorem}
\newtheorem{notation}[theorem]{Notation}
\newtheorem{exercise}[theorem]{Exercise}
\newcommand{\bex}{\begin{exercise}}
\newcommand{\ex}{\end{exercise}}

\newcommand{\pru}{{ \bfseries \textcolor{red}{Proof:} }}

%symbol definitions
\newcommand{\un}[1]{\underline{#1}}
\newcommand{\mbZ}{\mathbb{Z}}
\newcommand{\mbR}{\mathbb{R}}
\newcommand{\mbN}{\mathbb{N}}
\newcommand{\mbQ}{\mathbb{Q}}
\newcommand{\mbI}{\mathbb{I}}
\newcommand{\hra}{\hookrightarrow}
\newcommand{\tra}{\twoheadrightarrow}
\newcommand{\lra}{\leftrightarrow}
\newcommand{\ep}{\epsilon}
\newcommand{\Ra}{\Rightarrow}
\newcommand{\mb}[1]{\mathbb{#1}}
\newcommand{\mc}[1]{\mathcal{#1}}
\newcommand{\bfs}[1]{{\bfseries #1}}
\newcommand{\bs}[1]{\boldsymbol{#1}}
%Operator definitions
\DeclareMathOperator{\Irr}{Irr}
\DeclareMathOperator{\triv}{triv}
\DeclareMathOperator{\cyc}{cyc}
\DeclareMathOperator{\lcm}{lcm}
\DeclareMathOperator{\expo}{x}
\DeclareMathOperator{\ord}{o}
\DeclareMathOperator{\imm}{im}
\DeclareMathOperator{\sgn}{sgn}
\DeclareMathOperator{\Sym}{Sym}
\DeclareMathOperator{\alt}{alt}
\DeclareMathOperator{\irr}{irr}
\DeclareMathOperator{\eqt}{Equiv}
\DeclareMathOperator{\pat}{Part}
%\DeclareMathOperator{\sgn}{sgn}
%\DeclareMathOperator{\Aut}{Aut}
\DeclareMathOperator{\Gl}{Gl}
\DeclareMathOperator{\M}{M}
\DeclareMathOperator{\Id}{Id}
\DeclareMathOperator{\fixx}{Fix}
\DeclareMathOperator{\suppp}{Supp}
\DeclareMathOperator{\gl}{Gl}
\DeclareMathOperator{\id}{Id}
\DeclareMathOperator{\Aut}{Aut}
\DeclareMathOperator{\Inn}{Inn}
\DeclareMathOperator{\orb}{orb}
\DeclareMathOperator{\ii}{I}
\DeclareMathOperator{\im}{im}
\DeclareMathOperator{\Fix}{Fix}
\DeclareMathOperator{\Co}{Co}
\DeclareMathOperator{\md}{md}
\DeclareMathOperator{\qt}{qt}
\DeclareMathOperator{\ExtendedGCD}{ExtendedGCD}
\DeclareMathOperator{\Mod}{Mod}
\DeclareMathOperator{\GCD}{GCD}
\newcommand{\nms}{\negmedspace}
\newcommand{\nts}{\negthinspace}

\newcommand{\itep}{\item {\bfseries Problem}\ }
\newcommand{\gen}[1]{\langle \nts#1 \nts\rangle}
\newcommand{\quot}[2]{#1/ #2}
\newcommand{\order}[1]{\left|<\nts #1 \nts s>\right|}

%These next two commands are for making answers. 
\newcommand{\beans}{\begin{description} \item[{ \bfseries \textcolor{red}{Answer}}]\ }
\newcommand{\eans }{\end{description}}
%\newcommand{\begin{comment}ex}{{ \bfseries \textcolor{red}{Answer}}}

%To turn the answer into problem sets use replace to replace \begin{comment} with \begin{comment} and \\end{comment}  by \end{comment}.
\newcommand{\lieb}[3][{{}}]{\frac{d^#1 #2}{d\,#3^#1}}

\title{\textbf{Math 521 - Problem Set 2}}
\author{Guy Matz}
\date{\today}

\begin{document} 

%\maketitle
%\newpage % Q1

\begin{enumerate}[series=p]
\itep 34 \\
A map $f:M \to N$ is \textbf{open} if for each open set $U \subset M$ , the image set $f(U)$ is open in $N$.
\\
\benu
\item If $f$ is open, is it continuous?
\item If $f$ is a homeomorphism, is it open?
\item If $f$ is an open, continuous bijection, is it a homeomorphism?
\item If $f : \mbR \to \mbR$ is a continuous surjection, must it be open? 
\item If $f : \mbR \to \mbR$ is a continuous surjection, must it be a homeomorphism? 
\eenu
\textbf{Definition: Openness}\\
	A set $s \subset M$ is \textbf{open in M} if for every point $p \in S$ there is an $r \in \mbR$, $r>0$, so that for every $q \in M$,
	if $d(p,q) < r$, then $q < S$
	\begin{center}
		OR
		\\
		$\forall p \in S \, \exists r > 0 (B_r^M(p) \subseteq S)$
	\end{center}
\textbf{Definition - Continuous I}
	A function $f:M \to N$ is \textbf{continuous} if whenever $(a_i)$ is a sequence in $M$ and
	$$(a_i) \to b \text{ in } M,$$
	$$f((a_i)) \to f(b) \text{ in } N$$


\textbf{Definition - Continuous II }
	A function $f:M \to N$ is \textbf{continuous} if it satisfies the $\boldsymbol{\epsilon, \delta}$ \textbf{condition}:
	$\forall \epsilon > 0 \text{ and } \forall p \in M \, \exists \delta > 0 \text{ such that }$:
	$$q \in M \text{and } d(p,q) < \delta \implies d(fp,fq) < \epsilon$$

\textbf{Definition - Continuous III }
	A function $f:M \to N$ is \textbf{continuous} if it satisfies the $\boldsymbol{\epsilon, \delta}$ \textbf{condition}:
	$\forall \epsilon > 0 \text{ and } \forall p \in M \, \exists \delta > 0 \text{ such that }$:
	$$q \in M \text{and } d(p,q) < \delta \implies d(fp,fq) < \epsilon$$


\textbf{Definition - Continuous IV - Continu-ish}
	A function $f:M \to N$ is \textbf{continu-ish} if
	$\forall x \in M \forall \epsilon > 0 \exists \delta > 0 \forall y \in M$,\\
	$$d_M(x,y) < \delta \implies d_N(f(x), f(y)) < \epsilon$$

\textbf{Definition - Homeomorphism}
	If $f: M \to N$ is a bijection and $f$ is continuous and the inverse bijection $f^{-1}:N \to M$ is also continuous then $f$ is a \textbf{homeomorphism}
	($M \cong N$)
\newpage

\itep  14\\
The \textbf{distance} from a point $p$ in a metric space $M$ to a non-empty subset $S \subset M$ is defined to be dist$(p, S)$ = inf$\{d(p,s) : s \in S\}$
\\
\benu
\item Show that $p$ is a limit of $S$ if and only if dist$(p, S) = 0$
\item Show that $p \mapsto \text{dist}(p, S)$ is a uniformly continuous function of $p \in M$
\eenu
\textbf{Definition: inf - Infimum}\\
If $S$ is a non-empty subset of $\mbR$ then its Infimum is its greatest lower bound when $S$ is bounded below and is said to be $-\infty$ otherwise\\
\textbf{Definition: Metric Space}\\
A \textbf{metric space} is a pair $(M, d)$, where $M$ is a set and $d$ is a distance function on pairs of $M$, so that:
$$d: M^2 \to \mbR$$
\benu
\item Positive Definiteness: $d(x,y) \geq 0$ and $d(x,y) = 0$ iff $x = y$
\item Symmetry: $d(x,y) = d(y,x)$
\item Triangle Inequality: $d(x,z) \leq d(x,y) + d(y,z)$
\eenu
\textbf{Definition: Limit Point}\\
	For $S \subseteq M, p \in M, p$ is a \textbf{Limit point} of $S$ if there is a sequence $(a_i)$ so that $(a_i) \to p$ and $(a_i) \in S$
$$lim^M(S) = \{q \in M|q \text{ is a limit of }S\}$$

\newpage

\itep 25\\
Assume that $N$ is an open metric subspace of $M$ and that $U \subset N$
\benu
\item Prove that $U$ is open in $N$ if and only if it is open in $M$
\item Conversely, prove that if openness of $S \subset N$ is equivalent to openness in $M$ then $N$ is open in $M$
\item Do the same for closedness
\item Deduce that a clopen metric subspace $MN$ is the only example in which the concepts of openness and closedness in the subspace agree exactly with the concepts of the big space
\eenu
\textbf{Definition: Closedness}\\
A closed set contains all of it limits\\
\textbf{Definition: Limit}\\
A point $p \in M$ is a limit of $S$ if there exists a sequence $(p_n)$ in $S$ that converges to it.\\
\textbf{Corollary: p. 61}\\
A finite union of closed sets is closed
\\\\
A single point $p \in S$ can only form the sequence $(p, p, p, \dots)$ so its limit is $p$ and therefore it is closed.\\
\\
By the result above and the corollary stated above, a finite set of points is a closed set


\end{enumerate}
\end{document}