\documentclass[12pt]{amsart}
\setlength{\parskip}{.1in}
\setlength{\parindent}{0cm}
%myalterations
\usepackage{amssymb}
\usepackage{amsmath}
\usepackage[usenames,dvipsnames,svgnames,table]{xcolor}
\usepackage[colorlinks=true,urlcolor=blue,pdfborder={0 0 .5}pdfnewwindow=true]{hyperref}
\usepackage{enumitem}
%\usepackage{amsthm}
\usepackage{graphicx}
\usepackage{verbatim}
\usepackage{tabularx}
%\usepackage{arydshln,leftidx,mathtools}
\usepackage{bm}
\usepackage{tikz}
\usepackage{tikz-cd}
\usepackage{hyperref}
\usepackage{bm}
\usepackage{stmaryrd}

%\setlength{\dashlinedash}{.4pt}
%\setlength{\dashlinegap}{.8pt}
%\usepackage{amsthm}
\usepackage{verbatim}
%\usepackage{commath}
%My commands
%environment abbreviations
\newcommand{\benu}{\begin{enumerate}}
\newcommand{\eenu}{\end{enumerate}}
\newcommand{\bed}{\begin{description}}
\newcommand{\ed}{\end{description}}
\theoremstyle{definition}
\newtheorem{theorem}{Theorem}
\newtheorem{notation}[theorem]{Notation}
\newtheorem{exercise}[theorem]{Exercise}
\newcommand{\bex}{\begin{exercise}}
\newcommand{\ex}{\end{exercise}}

\newcommand{\pru}{{ \bfseries \textcolor{red}{Proof:} }}

%symbol definitions
\newcommand{\un}[1]{\underline{#1}}
\newcommand{\mbZ}{\mathbb{Z}}
\newcommand{\mbR}{\mathbb{R}}
\newcommand{\mbN}{\mathbb{N}}
\newcommand{\mbQ}{\mathbb{Q}}
\newcommand{\mbC}{\mathbb{C}}
\newcommand{\mbF}{\mathbb{F}}
\newcommand{\mcS}{\mathcal{S}}
\newcommand{\mcP}{\mathcal{P}}
\newcommand{\hra}{\hookrightarrow}
\newcommand{\tra}{\twoheadrightarrow}
\newcommand{\lra}{\leftrightarrow}
\newcommand{\ep}{\epsilon}
\newcommand{\Ra}{\Rightarrow}
\newcommand{\mb}[1]{\mathbb{#1}}
\newcommand{\mc}[1]{\mathcal{#1}}
\newcommand{\bfs}[1]{{\bfseries #1}}
\newcommand{\bs}[1]{\boldsymbol{#1}}
%Operator definitions
\DeclareMathOperator{\Irr}{Irr}
\DeclareMathOperator{\triv}{triv}
\DeclareMathOperator{\cyc}{cyc}
\DeclareMathOperator{\lcm}{lcm}
\DeclareMathOperator{\expo}{x}
\DeclareMathOperator{\ord}{o}
\DeclareMathOperator{\imm}{im}
\DeclareMathOperator{\sgn}{sgn}
\DeclareMathOperator{\Sym}{Sym}
\DeclareMathOperator{\alt}{alt}
\DeclareMathOperator{\irr}{irr}
\DeclareMathOperator{\eqt}{Equiv}
\DeclareMathOperator{\pat}{Part}
%\DeclareMathOperator{\sgn}{sgn}
%\DeclareMathOperator{\Aut}{Aut}
\DeclareMathOperator{\Gl}{Gl}
\DeclareMathOperator{\M}{M}
\DeclareMathOperator{\Id}{Id}
\DeclareMathOperator{\fixx}{Fix}
\DeclareMathOperator{\suppp}{Supp}
\DeclareMathOperator{\gl}{Gl}
\DeclareMathOperator{\id}{Id}
\DeclareMathOperator{\Aut}{Aut}
\DeclareMathOperator{\Inn}{Inn}
\DeclareMathOperator{\orb}{orb}
\DeclareMathOperator{\ii}{I}
\DeclareMathOperator{\im}{im}
\DeclareMathOperator{\Fix}{Fix}
\DeclareMathOperator{\Co}{Co}
\DeclareMathOperator{\md}{md}
\DeclareMathOperator{\qt}{qt}
\DeclareMathOperator{\ExtendedGCD}{ExtendedGCD}
\DeclareMathOperator{\Mod}{Mod}
\DeclareMathOperator{\GCD}{GCD}
\newcommand{\nms}{\negmedspace}
\newcommand{\nts}{\negthinspace}

\newcommand{\itep}{\item {\bfseries Problem}\ }
\newcommand{\gen}[1]{\langle \nts#1 \nts\rangle}
\newcommand{\quot}[2]{#1/ #2}
\newcommand{\order}[1]{\left|<\nts #1 \nts s>\right|}

%These next two commands are for making answers. 
\newcommand{\beans}{\begin{description} \item[{ \bfseries \textcolor{red}{Answer}}]\ }
\newcommand{\eans }{\end{description}}
%\newcommand{\begin{comment}ex}{{ \bfseries \textcolor{red}{Answer}}}

%To turn the answer into problem sets use replace to replace \begin{comment} with \begin{comment} and \\end{comment}  by \end{comment}.
\newcommand{\lieb}[3][{{}}]{\frac{d^#1 #2}{d\,#3^#1}}

\title{\textbf{Math 521 - Problem Set 5}}
\author{Guy Matz}
\date{\today}

\begin{document} 

\maketitle
\newpage % Q1

\begin{enumerate}[series=p]
\itep 6\\
If $S, T \subset M$, a metric space, and $S \subset T$, prove that
	\benu
	\item $\bar{S} \subset \bar{T}$\\
	$\bar{S}$ contains the limit points of $S$, so there is a sequence of elements in $S$ that converge to each limit in $S$.  Since $S \subset T$, these are sequences in $T$ whose limits points are in $\bar{T}$, hence $\bar{S} \subset \bar{T}$
	\item int$(S) \subset$ int($T$) \\
	Let $x \in $ int($S$).  We want to show that $x \in $ int(T).  We know that $S \subset T$, int(S)$ \subset S$ and  int(T) $\subset T$.  Since $S \subset T$, we have $x \in S \implies x \in T$ and $B_r^Sx \subset S$, so $B_r^Tx \subset T$.  Hence int(S) $\subset$ int(T).
	\eenu
\newpage

\itep 10\\
\benu
	\item Find a metric space in which the boundary of $M_rp$ is not equal to the sphere of radius $r$ at $p, \{x \in M : d(x,p) = r\}$\\
	\begin{description}
		\item[$M_rp$] $\{q \in M: d(p,q) < r\}$
	\end{description}
	The discrete metric is a metric in which the boundary of $M_rp$ is not equal to the sphere of radius $r$ at $p$.  If we choose a $p \in M$, then $M_1p = \{x \in M: d(x,p)<1\} = \{p\}$.  The sphere of radius 1, on the other hand, is $\{x \in M : d(x,p)=1 \} = M \setminus \{ p \}$.  The boundary is the intersection of these two sets, so $\partial M_1p = \overline{M_1p} \cap \overline{(M_1p)^c}$, which equals $\overline{\{p\}} \cap \overline{M \setminus \{p\}}$.  Since these two sets are closed, this is equal to $\{p\} \cap M \setminus \{p\} = \emptyset$.  Since the intersection is empty the two are not equal.
	\\
	\item Need the boundary be contained in the sphere?\\
	Yes.  
	
\eenu

\newpage

\itep \\
Prove that if $A$ and $B$ are compact, disjoint, non-empty subsets of $M$, then there are points $a\in A$ and $b\in B$ so that for any $x\in A$ and $y\in B$, $d(a,b)\leq d(x,y)$. In other words, $a$ and $b$ are the closest you get a point in $A$ and a point in $B$ to each other.
\\
\\
\textbf{Theorem:}
If $f:M \to \mbR$, $M$ is compact and $f$ is continuous, then $f$ attains a max and min value.
\\\\
$A$ and $B$ are both compact, so $A \times B$ is compact.  Since the distance function $d: A \times B \to \mbR$ is continuous, it follows from the theorem above that the distance between $A$ and $B$ attains a minimum value.

\newpage

\itep 28\\
Prove that there is an embedding of the line as a closed subset of the plane, and there is an embedding of the line as a bounded subset of the plane, but there is no embedding of the line as a closed and bounded subset of the plane

	\benu
		\item The constant function $f: \mbR \to \mbR^2$ defined by $f(x) = (x, c)$.  This is an embedding since $f \llbracket M \rrbracket$ is a homeomorphism since it is bijective and bi-continuous.  This subset of $\mbR^2$ is closed since its complement is open.
		\item The mapping $f: (0,1) \to \mbR^2$ defined by $f(x) = (x, 0)$ is a homeomorphism and the image is bounded subset of the plane.  Furthermore, $(0,1)$ is homeomorphic to $\mbR$.
		\item The closed and bounded subset of $\mbR^2$ is compact, however $\mbR$ is not.  Thus there cannot be an embedding since the two are not homeomorphic.
	\eenu
\newpage

\itep 43 \\
Suppose that $(K_n)$ is a nested sequence of compact non-empty sets, $K_1 \supset K_2 \supset \dots $, and $K = \bigcap K_n$.  If for some $\mu > 0$, each diam$K_n \geq \mu$, is it true that diam $K \geq \mu$.
\newpage

\itep 14\\
The \textbf{distance} from a point $p$ in a metric space $M$ to a non-empty subset $S \subset M$ is defined to be dist$(p, S) = \text{inf}\{d(p,s): s \in S\}$.
\benu
\item Show that $p$ is a limit of $S$ if and only if dist$(p, S) = 0$\\ % 21:00
\\
$\Rightarrow$  Assuming $p$ is a limit of $S$, we want to show dist$(p,S) = 0$.  Since $p$ is a limit point of $S$, there is a sequence $(a_i)$ such that $(a_i) \to p$ and $(a_i) \in S$
\\
$\Leftarrow$  Assuming dist$(p,S) = 0$, we want to show that $p$ is a limit of $S$.  Since dist$(p, S) = 0$, there exists some sequence that converges to the infimum.  So $d_i \to 0$ where $d_i \in \{d(p, s): s \in S\}$.  For each $d_i$ there is $s_i \in S$ such that $d(s_i, p) = d_i$.  Since $d_i \to 0, s_i \to p$.  
%Now, we have that $0 \leq d(p,S) \leq d(p, a_i)$. Using the squeeze theorem we have that $0 %\leq d(p,S)= \text{lim } d(p, a_i) \leq d(p, a_i)$.  
Hence $p$ is a limit of $S$.
\\
\item Show that $p \mapsto \text{dist}(p, S)$ is a uniformly continuous function of $p \in M$ % 30:00
\\
\begin{comment}

We want to show that $\forall \ep > 0 \, \exists \, \delta > 0 \, \forall x,y \in M$ such that $d(x,y) < \delta \implies |\text{dist}(x,s) - \text{dist}(y,s) < \ep|$\\
\end{comment}
We want to show that $|f(p)-f(q)|<d(p,q)$.  By The Triangle Inequality, $d(p,S) \leq d(p,q)+d(q,S)$.  Given $p,q \in M$, WLOG $d(p,S) - d(q,S) \geq 0$.


Now, notice that (using the WLOG in the first equality and the observation in the inequality)


$|d(p,S)-d(q,S)|=d(p,S)-d(q,S) \leq d(p,q)+d(q,S)-d(q,S)=d(p,q)$.


Therefore, $|f(p)-f(q)|=|d(p,S)-d(q,S)| \leq d(p,q)$.


This implies uniformly continuity since given $\epsilon$, we can take $\delta = \epsilon$. Therefore, if $d(p,q)<\delta = \epsilon$ then $|f(p)-f(q)|<d(p,q)<\epsilon$.

\eenu
\newpage

\itep 41\\
Prove that the 2-sphere is not homeomorphic to the plane\\
\\
The 2-sphere is closed and bounded in $\mbR$ so it is compact.  The plane, however, is not compact, so the two cannot be homeomorphic.

\newpage
\itep 54\\
If $S$ is connected, is the interior of $S$ connected?  Prove this or give a counter-example
\\\\
The plane is a counter-example since it is connected but its interior is not.
\newpage

\itep 55\\
Theorem 51 states that the closure of a connected set is connected.
	\benu
		\item Is the closure of a disconnected set disconnected\\
		No.  The closure of $\mbQ$, a disconnected set, is $\mbR$, a connected set.
		\item What about the interior of a disconnected set?\\
		No.  The set $[0,1] \cup \{2\}$ is disconnected , but its interior $(0,1)$ is connected.
	\eenu
\newpage

\itep 56\\
Prove that every countable metric space (not empty and not a singleton) is disconnected.  [Astonishingly, there exists a countable topological space which is connected.  Its topology does not arise from a metric]

\newpage
\itep 59\\
Prove that the annulus $A = \{z \in \mbR^2: r \leq |z| \leq R\}$ is connected.
\\
Th annulus is a 2-dimensional ring-shaped object.  Any 2 points can be path-connected  - and therefore connected - by a continuous function be determining an arc between the two points of appropriate radius and center.
\\
Let $x$ and $y$ be two points in the annulus.  If they are in the same quadrant, fix a point $p$ at $(1 + n) \pi /4$ on the outer boundary of the annulus, where $n$ is the quadrant of the annulus they are in.  The two points $x, y$ can connected to $p$ by a straight line so they are path-connected to $p$ and therefore connected.  Otherwise, fix two points $p,q$ on the outer boundary of the annulus which can be connected with a straight line so that $x$ can connect to $q$ by a straight line and $y$ and connect to $q$ by a straight line.  The two points $x, y$ can connected to $p, q$, respectively, by a straight line so they are path-connected and therefore connected.  
\newpage
\itep 64a\\
Prove that every connected open subset of $\mbR^m$ is a path connected.
\\\\
We want to show that if we have a connected open subset of $\mbR$ then the subset if path connected.
\\
\textbf{Connected:}  $M$ is \textbf{connected} iff it is not the union of two non-empty, disjoint open sets.
\\
By the definition above, the subsets have a non-empty intersection, so  . . .  
\newpage
\itep 68\\
Prove that $(a,b)$ and $[a,b]$ are not homeomorphic metric spaces
\\
\\
$[a,b]$ is a compact space, but $(a,b)$ is not.  Therefore there is no continuous function $f : [a,b] \to (a,b)$ so the two cannot be homeomorphic.

\newpage

\itep 69\\
Let $M$ and $N$ be non-empty metric spaces
	\benu
		\item If $M$ and $N$ are connected, prove that $M \times N$ is connected
		Suppose towards a contradiction that $M, N$ connected imples $M \times N$ is disconnected.  If $M \times N$ is disconnected, then it contains a clopen subset in either $M$ or $N$, but they are both connected.
		\item What about the converse?
		
		\item Answer the question again for path-connectedness
		Suppose towards a contradiction that $M, N$ path-connected imples $M \times N$ is not path-connected.  If $M \times N$ is not path-connected, then there does not exist a continuous function that joins every point in $M \times N$ to another.  So then either $M$ or $N$ is not path-connected.  But hey are both path-connected.
	\eenu
\newpage


\itep 74\\
	\benu
		\item The intersection of connected sets need not be connected.  Give an example.
		\\
		A circle and a line are both connected sets, however the intersection of the line across the diameter of the circle is the set os two disconnected points.
		\item Suppose that $S_1, S_2, S_3, \dots$ is a sequence of connected, closed subsets of the plane and $S_1 \supset S_2 \supset \dots$.  Is $S = \bigcap S_n$ connected?
		\item * Does the answer change if the sets are compact
		\item What is the situation for a nested decreasing sequence of compact path-connected sets?
	\eenu
\newpage


\itep 77\\
If a metric space $M$ is the union of path-connected sets $S_\alpha$, all of which have the path-connected set $K$ in common, is $M$ path-connected?\\
\\
Let $p \in K$ and $a,b \in \bigcup_{\alpha \in \mbZ}S_\alpha$.  Then there exists $\alpha_1$ and $\alpha_2$ such that $a \in S_{\alpha_1}$ and $S_{\alpha_2}$.  As well, $K \cap S_{\alpha_1} \neq \emptyset$ and $K \cap S_{\alpha_2} \neq \emptyset$.  Now let $f$ be a continuous function that describes a path from $a$ to $p$, and $g$ a continuous function that describes a path from $p$ to $b$.  Then, since the composition of continuous functions is continuous, there exists a continuous function that connects $a$ and $b$, hence $M$ is path-connected.


\end{enumerate}
\end{document}
