\documentclass[12pt]{amsart}
\setlength{\parskip}{.1in}
\setlength{\parindent}{0cm}
%myalterations
\usepackage{amssymb}
\usepackage{amsmath}
\usepackage[usenames,dvipsnames,svgnames,table]{xcolor}
\usepackage[colorlinks=true,urlcolor=blue,pdfborder={0 0 .5}pdfnewwindow=true]{hyperref}
\usepackage{enumitem}
%\usepackage{amsthm}
\usepackage{graphicx}
\usepackage{verbatim}
\usepackage{tabularx}
%\usepackage{arydshln,leftidx,mathtools}
\usepackage{bm}
\usepackage{tikz}
\usepackage{tikz-cd}
\usepackage{hyperref}
\usepackage{bm}

%\setlength{\dashlinedash}{.4pt}
%\setlength{\dashlinegap}{.8pt}
%\usepackage{amsthm}
\usepackage{verbatim}
%\usepackage{commath}
%My commands
%environment abbreviations
\newcommand{\benu}{\begin{enumerate}}
\newcommand{\eenu}{\end{enumerate}}
\newcommand{\bed}{\begin{description}}
\newcommand{\ed}{\end{description}}
\theoremstyle{definition}
\newtheorem{theorem}{Theorem}
\newtheorem{notation}[theorem]{Notation}
\newtheorem{exercise}[theorem]{Exercise}
\newcommand{\bex}{\begin{exercise}}
\newcommand{\ex}{\end{exercise}}

\newcommand{\pru}{{ \bfseries \textcolor{red}{Proof:} }}

%symbol definitions
\newcommand{\un}[1]{\underline{#1}}
\newcommand{\mbZ}{\mathbb{Z}}
\newcommand{\mbR}{\mathbb{R}}
\newcommand{\mbN}{\mathbb{N}}
\newcommand{\mbQ}{\mathbb{Q}}
\newcommand{\mbC}{\mathbb{C}}
\newcommand{\mbF}{\mathbb{F}}
\newcommand{\mcS}{\mathcal{S}}
\newcommand{\mcP}{\mathcal{P}}
\newcommand{\hra}{\hookrightarrow}
\newcommand{\tra}{\twoheadrightarrow}
\newcommand{\lra}{\leftrightarrow}
\newcommand{\ep}{\epsilon}
\newcommand{\Ra}{\Rightarrow}
\newcommand{\mb}[1]{\mathbb{#1}}
\newcommand{\mc}[1]{\mathcal{#1}}
\newcommand{\bfs}[1]{{\bfseries #1}}
\newcommand{\bs}[1]{\boldsymbol{#1}}
%Operator definitions
\DeclareMathOperator{\Irr}{Irr}
\DeclareMathOperator{\triv}{triv}
\DeclareMathOperator{\cyc}{cyc}
\DeclareMathOperator{\lcm}{lcm}
\DeclareMathOperator{\expo}{x}
\DeclareMathOperator{\ord}{o}
\DeclareMathOperator{\imm}{im}
\DeclareMathOperator{\sgn}{sgn}
\DeclareMathOperator{\Sym}{Sym}
\DeclareMathOperator{\alt}{alt}
\DeclareMathOperator{\irr}{irr}
\DeclareMathOperator{\eqt}{Equiv}
\DeclareMathOperator{\pat}{Part}
%\DeclareMathOperator{\sgn}{sgn}
%\DeclareMathOperator{\Aut}{Aut}
\DeclareMathOperator{\Gl}{Gl}
\DeclareMathOperator{\M}{M}
\DeclareMathOperator{\Id}{Id}
\DeclareMathOperator{\fixx}{Fix}
\DeclareMathOperator{\suppp}{Supp}
\DeclareMathOperator{\gl}{Gl}
\DeclareMathOperator{\id}{Id}
\DeclareMathOperator{\Aut}{Aut}
\DeclareMathOperator{\Inn}{Inn}
\DeclareMathOperator{\orb}{orb}
\DeclareMathOperator{\ii}{I}
\DeclareMathOperator{\im}{im}
\DeclareMathOperator{\Fix}{Fix}
\DeclareMathOperator{\Co}{Co}
\DeclareMathOperator{\md}{md}
\DeclareMathOperator{\qt}{qt}
\DeclareMathOperator{\ExtendedGCD}{ExtendedGCD}
\DeclareMathOperator{\Mod}{Mod}
\DeclareMathOperator{\GCD}{GCD}
\newcommand{\nms}{\negmedspace}
\newcommand{\nts}{\negthinspace}

\newcommand{\itep}{\item {\bfseries Problem}\ }
\newcommand{\gen}[1]{\langle \nts#1 \nts\rangle}
\newcommand{\quot}[2]{#1/ #2}
\newcommand{\order}[1]{\left|<\nts #1 \nts s>\right|}

%These next two commands are for making answers. 
\newcommand{\beans}{\begin{description} \item[{ \bfseries \textcolor{red}{Answer}}]\ }
\newcommand{\eans }{\end{description}}
%\newcommand{\begin{comment}ex}{{ \bfseries \textcolor{red}{Answer}}}

%To turn the answer into problem sets use replace to replace \begin{comment} with \begin{comment} and \\end{comment}  by \end{comment}.
\newcommand{\lieb}[3][{{}}]{\frac{d^#1 #2}{d\,#3^#1}}

\title{\textbf{Math 521 - Problem Set 5}}
\author{Guy Matz}
\date{\today}

\begin{document} 

%\maketitle
%\newpage % Q1

\begin{enumerate}[series=p]
\itep 6\\
If $S, T \subset M$, a metric space, and $S \subset T$, prove that
	\benu
	\item $\bar{S} \subset \bar{T}$
	\item int$(S) \subset$ int($T$) 
	\eenu
\newpage

\itep 10\\
\benu
	\item Find a metric space in which the boundary of $M_rp$ is not equal to the sphere of radius $r$ at $p, \{x \in M : d(x,p) = r\}$
	\item Need the boundary be contained in the sphere?
\eenu

\newpage

\itep \\
Prove that if $A$ and $B$ are compact, disjoint, non-empty subsets of $M$, then there are points $a\in A$ and $b\in B$ so that for any $x\in A$ and $y\in B$, $d(a,b)\leq d(x,y)$. In other words, $a$ and $b$ are the closest you get a point in $A$ and a point in $B$ to each other.

\newpage

\itep 28\\
Prove that there is an embedding of the line as a closed subset of the plane, and there is an embedding of the line as a bounded subset of the plane, but there is no embedding of the line as a closed and bounded subset of the plane

\newpage

\itep 43 \\
Suppose that $(K_n)$ is a nested sequence of compact non-empty sets, $K_1 \supset K_2 \supset \dots $, and $K = \bigcap K_n$.  If for some $\mu > 0$, each diam$K_n \geq \mu$, is it true that diam $K \geq \mu$.
\newpage

\itep 14\\
The \textbf{distance} from a point $p$ in a metric space $M$ to a non-empty subset $S \subset M$ is defined to be dist$(p, S) = \text{inf}\{d(p,s): s \in S\}$.
	\benu
		\item Show that $p$ is a limit of $S$ if and only if dist$(p,S) = 0$
		\item Show that $p \mapsto \text{dist}(p, S)$ is a uniformly continuous function of $p \in M$
	\eenu
\newpage

\itep 41\\
Prove that the 2-sphere is not homeomorphic to the plane

\newpage
\itep 54\\
If $S$ is connected, is the interior of $S$ connected?  Prove this or give a counter-example

\newpage

\itep 55\\
Theorem 51 states that the closure of a connected set is connected.
	\benu
		\item Is the closure of a disconnected set disconnected
		\item What about the interior of a disconnected set?
	\eenu
\newpage

\itep 56\\
Prove that every countable metric space (not empty and not a singleton) is disconnected.  [Astonishingly, there exists a countable topological space which is connected.  Its topology does not arise from a metric]

\newpage
\itep 59\\
Prove that the annulus $A = \{z \in \mbR^2: r \leq |z| \leq R\}$ is connected.

\newpage
\itep 64a\\
Prove that every connected open subset of $\mbR^m$ is a path connected.

\newpage
\itep 68\\
Prove that $(a,b)$ and $[a,b]$ are not homeomorphic metric spaces


\newpage

\itep 69\\
Let $M$ and $N$ be non-empty metric spaces
	\benu
		\item If $M$ and $N$ are connected, prove that $M \times N$ is connected
		\item What about the converse?
		\item Answer the question again for path-connectedness
	\eenu
\newpage


\itep 74\\
	\benu
		\item The intersection of connected sets need not be connected.  Give an example.
		\item Suppose that $S_1, S_2, S_3, \dots$ is a sequence of connected, closed subsets of the plane and $S_1 \supset S_2 \supset \dots$.  Is $S = \bigcap S_n$ connected?
		\item * Does the answer change if the sets are compact
		\item What is the situation for a nested decreasing sequence of compact path-connected sets?
	\eenu
\newpage


\itep 77\\
If a metric space $M$ is the union of path-connected sets $S_\alpha$, all of which have the path-connected set $K$ in common, is $M$ path-connected?



\end{enumerate}
\end{document}
