\documentclass[12pt]{amsart}
\setlength{\parskip}{.1in}
\setlength{\parindent}{0cm}
%myalterations
\usepackage{amssymb}
\usepackage{amsmath}
\usepackage[usenames,dvipsnames,svgnames,table]{xcolor}
\usepackage[colorlinks=true,urlcolor=blue,pdfborder={0 0 .5}pdfnewwindow=true]{hyperref}
\usepackage{enumitem}
%\usepackage{amsthm}
\usepackage{graphicx}
\usepackage{verbatim}
\usepackage{tabularx}
%\usepackage{arydshln,leftidx,mathtools}
\usepackage{bm}
\usepackage{tikz}
\usepackage{tikz-cd}
\usepackage{hyperref}
\usepackage{bm}

%\setlength{\dashlinedash}{.4pt}
%\setlength{\dashlinegap}{.8pt}
%\usepackage{amsthm}
\usepackage{verbatim}
%\usepackage{commath}
%My commands
%environment abbreviations
\newcommand{\benu}{\begin{enumerate}}
\newcommand{\eenu}{\end{enumerate}}
\newcommand{\bed}{\begin{description}}
\newcommand{\ed}{\end{description}}
\theoremstyle{definition}
\newtheorem{theorem}{Theorem}
\newtheorem{notation}[theorem]{Notation}
\newtheorem{exercise}[theorem]{Exercise}
\newcommand{\bex}{\begin{exercise}}
\newcommand{\ex}{\end{exercise}}

\newcommand{\pru}{{ \bfseries \textcolor{red}{Proof:} }}

%symbol definitions
\newcommand{\un}[1]{\underline{#1}}
\newcommand{\mbZ}{\mathbb{Z}}
\newcommand{\mbR}{\mathbb{R}}
\newcommand{\mbN}{\mathbb{N}}
\newcommand{\mbQ}{\mathbb{Q}}
\newcommand{\mbC}{\mathbb{C}}
\newcommand{\mbF}{\mathbb{F}}
\newcommand{\mcS}{\mathcal{S}}
\newcommand{\mcP}{\mathcal{P}}
\newcommand{\mcR}{\mathcal{R}}
\newcommand{\hra}{\hookrightarrow}
\newcommand{\tra}{\twoheadrightarrow}
\newcommand{\lra}{\leftrightarrow}
\newcommand{\ep}{\epsilon}
\newcommand{\Ra}{\Rightarrow}
\newcommand{\mb}[1]{\mathbb{#1}}
\newcommand{\mc}[1]{\mathcal{#1}}
\newcommand{\bfs}[1]{{\bfseries #1}}
\newcommand{\bs}[1]{\boldsymbol{#1}}
%Operator definitions
\DeclareMathOperator{\Irr}{Irr}
\DeclareMathOperator{\triv}{triv}
\DeclareMathOperator{\cyc}{cyc}
\DeclareMathOperator{\lcm}{lcm}
\DeclareMathOperator{\expo}{x}
\DeclareMathOperator{\ord}{o}
\DeclareMathOperator{\imm}{im}
\DeclareMathOperator{\sgn}{sgn}
\DeclareMathOperator{\Sym}{Sym}
\DeclareMathOperator{\alt}{alt}
\DeclareMathOperator{\irr}{irr}
\DeclareMathOperator{\eqt}{Equiv}
\DeclareMathOperator{\pat}{Part}
%\DeclareMathOperator{\sgn}{sgn}
%\DeclareMathOperator{\Aut}{Aut}
\DeclareMathOperator{\Gl}{Gl}
\DeclareMathOperator{\M}{M}
\DeclareMathOperator{\Id}{Id}
\DeclareMathOperator{\fixx}{Fix}
\DeclareMathOperator{\suppp}{Supp}
\DeclareMathOperator{\gl}{Gl}
\DeclareMathOperator{\id}{Id}
\DeclareMathOperator{\Aut}{Aut}
\DeclareMathOperator{\Inn}{Inn}
\DeclareMathOperator{\orb}{orb}
\DeclareMathOperator{\ii}{I}
\DeclareMathOperator{\im}{im}
\DeclareMathOperator{\Fix}{Fix}
\DeclareMathOperator{\Co}{Co}
\DeclareMathOperator{\md}{md}
\DeclareMathOperator{\qt}{qt}
\DeclareMathOperator{\ExtendedGCD}{ExtendedGCD}
\DeclareMathOperator{\Mod}{Mod}
\DeclareMathOperator{\GCD}{GCD}
\newcommand{\nms}{\negmedspace}
\newcommand{\nts}{\negthinspace}

\newcommand{\itep}{\item {\bfseries Problem}\ }
\newcommand{\gen}[1]{\langle \nts#1 \nts\rangle}
\newcommand{\quot}[2]{#1/ #2}
\newcommand{\order}[1]{\left|<\nts #1 \nts s>\right|}

%These next two commands are for making answers. 
\newcommand{\beans}{\begin{description} \item[{ \bfseries \textcolor{red}{Answer}}]\ }
\newcommand{\eans }{\end{description}}
%\newcommand{\begin{comment}ex}{{ \bfseries \textcolor{red}{Answer}}}

%To turn the answer into problem sets use replace to replace \begin{comment} with \begin{comment} and \\end{comment}  by \end{comment}.
\newcommand{\lieb}[3][{{}}]{\frac{d^#1 #2}{d\,#3^#1}}

\title{\textbf{Math 521 - Problem Set 7}}
\author{Guy Matz}
\date{\today}

\begin{document} 

\maketitle
%\newpage % Q1

\begin{enumerate}[series=p]
\itep 1\\
Assume that $f : \mbR \to \mbR$ satisifies $|f(t) - f(x)| \leq |t - x|^2$ for all $t, x$.  Prove that $f$ is constant.

\newpage

\itep 2\\
A function satisfies a \textbf{H\"{o}lder condition of order $\alpha$} if $\alpha > 0$, and for some constant H and all $u, x \in (a,b)$,
$$|f(u) - f(x)| \leq H|u - x|^{\alpha}.$$
The function is said to be $\mathbf{\alpha}$\textbf{-H\"{o}lder}, with $\alpha$-H\"{o}lder constant $H$.  (The terms "Lipschitz function of order $\alpha$" and "$\alpha$-Lipschitz function" are sometimes used with the same meaning.)
	\benu
		\item Prove that an $\alpha$-H\"{o}lder function defined on $(a,b)$ is uniformly continuous and infer that it extends uniquely to a continuous function defined on $[a,b]$.  Is the extended function $\alpha$-H\"{o}lder?
		\item What does $\alpha$-H\"{o}lder continuity mean when $\alpha = 1$?
		\item Prove that $\alpha$-H\"{o}lder continuity when $\alpha> 1$ implies that $f$ is constant.
	\eenu
\newpage

\itep 3\\
Assume that $f:(a,b) \to \mbR$ is differentiable
	\benu
		\item If $f'(x) > 0$ for all $x$, prove that $f$ is strictly monotone increasing
		\item If $f'(x) \geq 0$ for all $x$, what can you prove?
	\eenu

\newpage

\itep 5\\
Assume that $f: \mbR \to \mbR$ is continuous, and for all $x \neq 0, f'(x)$ exists.  If $\lim\limits_{x \to 0} f'(x) = L$ exists, does it follow that $f'(0)$ exists?  Prove or disprove.

\newpage
\itep 7\\
In L'Hospitals Rule, replace the interval $(a,b)$ with the half-line $[a, \infty)$ and interpret "$x$ tends to $b$" as "$x \to \infty$."  Show that if $f/g$ tends to $0/0$ and $f'/g'$ tends to $L$ then $f/g$ also tends to $L$.  Prove that this continues to hold when $L = \infty$ in the sense that if $f'/g' \to \infty$ then $f/g \to \infty$.

\newpage
\itep 8\\
In L'Hospitals Rule, replace the assumption that $f/g$ tends to $0/0$ with the assumption that it tends to $\infty/\infty$.  If $f'/g'$ tends to $L$, prove that $f/g$ tends to $L$ also.  [Hint: Think of a rear guard instead of an advance guard.] [Query: Is there a way to deduce the $\infty/\infty$ case from the $0/0$ case?  Naively taking reciprocals does not work.]


\newpage
\itep 28\\
In many calculus books, the definition of the integral is given as:
$$\lim\limits_{n \to \infty} \sum_{k=1}^{n}f(x^*_k) \frac{b-a}{n}$$
where $x^*_k$ is the midpoint of the interval
$$[a+(k-1)(b-a)/n, a + k(b-a)/n].$$
	\benu
		\item If $f$ is continuous show that the calculus book limit exists and equals the Riemann integral of $f$. [Hint: This is a one-liner]
		\item Show by example that the calculus style limit can exist for functions which are not Riemann integrable
		\item Infer that the calculus style definition of the integral is inadequate for real analysis
	\eenu


\newpage
\itep 30\\
Prove that the interval $[0,1]$ is not a zero set.  [Hint: Be careful; this is not entirely trivial]



\newpage
\itep 32\\
Define a Cantor set by removing from $[0,1]$ the middle interval of length 1/4.  From the remaining two intervals $F^1$ remove the middle intervals of length 1/16.  From the remaining four intervals $F^2$ remove the middle intervals of length 1/64, and so on.  At the $n^{\text{th}}$ step in the construction $F^n$ consists of $2^n$ subintervals of $F^{n-1}$.
	\benu
		\item Prove that $F = \bigcap F^n$ is a Cantor set but not a zero set.  It is often referred to as a \textbf{fat Cantor set}
		\item Infer that being a zero set is not a topological property: if two sets are homeomorphic and one is a zero set, then the other need not be a zero set.
	\eenu
	
[Hint: To get a sense of this fat Cantor set, calculate the total length of the intervals which comprise its complement]

\newpage
\itep 34\\
	\benu
		\item Prove that the characteristic function $f$ of the middle-thirds Cantor set $C$ is Riemann integrable but the characteristic function $g$ of the fat Cantor set $F$ is not.
		\item Why is there a homeomorphism $h: [0,1] \to [0,1]$ sending $C$ onto $F$?
		\item Infer that the composite of Riemann integrable functions need not be Riemann integrable.  How is this example related to Corollaries 26, 30 of the Riemann-Lebesgue Theorem?
	\eenu

\newpage
\itep 50\\
If $f,g$ are Riemann integrable on $[a,b]$, and $f(x) < g(x)$ for all $x \in [a,b]$, prove that $\int_{a}^{b} f(x) dx < \int_{a}^{b} g(x) dx$.  (Note the strict inequality.)


\newpage
\itep 51\\
Let $f:[a,b] \to \mbR$ be given.  Prove or give counter-examples to the following assertions
	\benu
		\item $f \in \mbR \Rightarrow |f| \in \mbR$
		\item $|f| \in \mbR \Rightarrow f \in \mbR$
		\item $f \in \mbR \text{ and } |f(x)| \geq c >0 \text{ for all } x \Rightarrow 1/f \in \mbR$
		\item $f \in \mbR \Rightarrow f^2 \in \mbR$
		\item $f^2 \in \mbR \Rightarrow f \in \mbR$
		\item $f^3 \in \mbR \Rightarrow f \in \mbR$
		\item $f^2 \in \mbR \text{ and } f(x) \geq 0 \text{ for all } x \Rightarrow f \in \mcR$
	\eenu
[Here $f^2$ and $f^3$ refer to the functions $f(x) \cdot f(x)$ and $f(x) \cdot f(x) \cdot f(x)$, not the iterates.]


\newpage
\itep 52\\
Given $f, g \in \mcR$, prove that max($f,g$), min($f,g$) $\in \mcR$, where max($f,g)(x) = $ max$(f(x), g(x))$ and min($f,g)(x) = $ min$(f(x), g(x)).$



\newpage
\itep 53\\
Assume that $f,g:[0,1] \to \mbR$ are Riemann Integrable and $f(x) = g(x)$ except on the middle-thirds of the Cantor set $C$.
	\benu
		\item Prove that $f$ and $g$ have the same integral
		\item Is the same true if $f(x) = g(x)$ except as $x \in \mbQ$
		\item How is this related to the fact that the characteristic function of $\mbQ$ is not Riemann integrable?
	\eenu

\end{enumerate}
\end{document}
