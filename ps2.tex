\documentclass[12pt]{amsart}
\setlength{\parskip}{.1in}
\setlength{\parindent}{0cm}
%myalterations
\usepackage{amssymb}
\usepackage{amsmath}
\usepackage[usenames,dvipsnames,svgnames,table]{xcolor}
\usepackage[colorlinks=true,urlcolor=blue,pdfborder={0 0 .5}pdfnewwindow=true]{hyperref}
\usepackage{enumitem}
%\usepackage{amsthm}
\usepackage{graphicx}
\usepackage{verbatim}
\usepackage{tabularx}
%\usepackage{arydshln,leftidx,mathtools}
\usepackage{bm}
\usepackage{tikz}
\usepackage{tikz-cd}
\usepackage{hyperref}
\usepackage{bm}

%\setlength{\dashlinedash}{.4pt}
%\setlength{\dashlinegap}{.8pt}
%\usepackage{amsthm}
\usepackage{verbatim}
%\usepackage{commath}
%My commands
%environment abbreviations
\newcommand{\benu}{\begin{enumerate}}
\newcommand{\eenu}{\end{enumerate}}
\newcommand{\bed}{\begin{description}}
\newcommand{\ed}{\end{description}}
\theoremstyle{definition}
\newtheorem{theorem}{Theorem}
\newtheorem{notation}[theorem]{Notation}
\newtheorem{exercise}[theorem]{Exercise}
\newcommand{\bex}{\begin{exercise}}
\newcommand{\ex}{\end{exercise}}

\newcommand{\pru}{{ \bfseries \textcolor{red}{Proof:} }}

%symbol definitions
\newcommand{\un}[1]{\underline{#1}}
\newcommand{\mbZ}{\mathbb{Z}}
\newcommand{\mbR}{\mathbb{R}}
\newcommand{\mbN}{\mathbb{N}}
\newcommand{\mbQ}{\mathbb{Q}}
\newcommand{\mbC}{\mathbb{C}}
\newcommand{\mbF}{\mathbb{F}}
\newcommand{\mcS}{\mathcal{S}}
\newcommand{\mcP}{\mathcal{P}}
\newcommand{\hra}{\hookrightarrow}
\newcommand{\tra}{\twoheadrightarrow}
\newcommand{\lra}{\leftrightarrow}
\newcommand{\ep}{\epsilon}
\newcommand{\Ra}{\Rightarrow}
\newcommand{\mb}[1]{\mathbb{#1}}
\newcommand{\mc}[1]{\mathcal{#1}}
\newcommand{\bfs}[1]{{\bfseries #1}}
\newcommand{\bs}[1]{\boldsymbol{#1}}
%Operator definitions
\DeclareMathOperator{\Irr}{Irr}
\DeclareMathOperator{\triv}{triv}
\DeclareMathOperator{\cyc}{cyc}
\DeclareMathOperator{\lcm}{lcm}
\DeclareMathOperator{\expo}{x}
\DeclareMathOperator{\ord}{o}
\DeclareMathOperator{\imm}{im}
\DeclareMathOperator{\sgn}{sgn}
\DeclareMathOperator{\Sym}{Sym}
\DeclareMathOperator{\alt}{alt}
\DeclareMathOperator{\irr}{irr}
\DeclareMathOperator{\eqt}{Equiv}
\DeclareMathOperator{\pat}{Part}
%\DeclareMathOperator{\sgn}{sgn}
%\DeclareMathOperator{\Aut}{Aut}
\DeclareMathOperator{\Gl}{Gl}
\DeclareMathOperator{\M}{M}
\DeclareMathOperator{\Id}{Id}
\DeclareMathOperator{\fixx}{Fix}
\DeclareMathOperator{\suppp}{Supp}
\DeclareMathOperator{\gl}{Gl}
\DeclareMathOperator{\id}{Id}
\DeclareMathOperator{\Aut}{Aut}
\DeclareMathOperator{\Inn}{Inn}
\DeclareMathOperator{\orb}{orb}
\DeclareMathOperator{\ii}{I}
\DeclareMathOperator{\im}{im}
\DeclareMathOperator{\Fix}{Fix}
\DeclareMathOperator{\Co}{Co}
\DeclareMathOperator{\md}{md}
\DeclareMathOperator{\qt}{qt}
\DeclareMathOperator{\ExtendedGCD}{ExtendedGCD}
\DeclareMathOperator{\Mod}{Mod}
\DeclareMathOperator{\GCD}{GCD}
\newcommand{\nms}{\negmedspace}
\newcommand{\nts}{\negthinspace}

\newcommand{\itep}{\item {\bfseries Problem}\ }
\newcommand{\gen}[1]{\langle \nts#1 \nts\rangle}
\newcommand{\quot}[2]{#1/ #2}
\newcommand{\order}[1]{\left|<\nts #1 \nts s>\right|}

%These next two commands are for making answers. 
\newcommand{\beans}{\begin{description} \item[{ \bfseries \textcolor{red}{Answer}}]\ }
\newcommand{\eans }{\end{description}}
%\newcommand{\begin{comment}ex}{{ \bfseries \textcolor{red}{Answer}}}

%To turn the answer into problem sets use replace to replace \begin{comment} with \begin{comment} and \\end{comment}  by \end{comment}.
\newcommand{\lieb}[3][{{}}]{\frac{d^#1 #2}{d\,#3^#1}}

\title{\textbf{Math 521 - Problem Set 1}}
\author{Guy Matz}
\date{\today}

\begin{document} 

%\maketitle
%\newpage % Q1

\begin{enumerate}[series=p]
\itep 1 \\
$(0, 1)$ is an open subset of $\mbR$ but not of $\mbR^2$, when we think of $\mbR$ as the $x$-axis in $\mbR^2$.  Prove this.


\newpage

\itep 2 \\
For which intervals $[a,b]$ in $\mbR$ is the intersection $[a,b] \cap \mbQ$ a clopen subset of the metric space $\mbQ$?

\newpage

\itep 3\\
Prove directly from the definition of closed set that each single point is a closed subset of a metric space.  Why does this imply that a finite set of points is also a closed set?

\newpage


\itep 5\\
Prove that a set $U \subset M$ is open if and only if none of its points are limits of its complement

\newpage

\itep 11\\
Let $\mc{T}$ be the collection of open subsets of a metric space $M$, and $\mc{K}$ the collection of closed subsets.  Show that there is a bijection from $\mc{T}$ onto $\mc{K}$

\newpage

\itep 12\\
Let $M$ be the metric space with the discrete metric, or more generally a homemorph of $M$.
\benu
\item Prove that every subset of $M$ is clopen\\
\item Prove that every function defined on $M$ is continuous.\\
\item Which sequences converge in $M$?
\eenu

\newpage

\itep 17 \\
Assume that $f:M \rightarrow N$ is a function from one metric space to another which satisfies the following condition:  if a sequence $(p_n)$ in $M$ converges then the sequence $(f(p_n))$ in $N$ converges.  Prove that $f$ is continuous.

\newpage

\itep 18\\
The simplest type of mapping from one metric space to another is an $\textbf{isometry}$.  It is a bijection $f:M \rightarrow N$ that preserves distance in the sense that for all $p,q \in M$,
$$d_N(fp, fq) = d_M(p,q).$$
If there exists an isometry from $M$ to $N$ then $M$ and $N$ are said to be isometric, $M \equiv N$.  You might have two copies of a unit equilateral triangle, one centered at the origin and one centered elsewhere.  They are isometric.  Isometric metric spaces are indistinguishable as metric spaces.
\benu
\item Prove that every isometry is continuous
\item Prove that every isometry is a homeomorphism
\item Prove that $[0,1]$ is not isometric to $[0,2]$.
\eenu

\newpage

\itep 19\\
Prove that isometry is an equivalence relation: if $M$ is isometric to $N$, show that $N$ is isometric to $M$; show that any $M$ is isometric to itself (what mapping of $M$ to $N$ is isometry?); if $M$ is isometric to $N$ and $N$ is isometric to $P$, show that $M$ is isometric to $P$.

\newpage

\itep 20\\
Is the perimeter of a square isometric to the circle?  Homeomorphic?  Explain.

\newpage

\itep 22\\
Is $\mbR$ homemorphic to $\mbQ$?  Explain.

\newpage

\itep 23\\
Is $\mbQ$ homemorphic to $\mbN$?  Explain.

\newpage

\itep 29\\
\benu
\item Prove that every convergent sequence is bounded.  That is, if $(p_n)$ converges in the metric space $M$, prove that there is some neighborhood $M_rq$ containing the set $\{p_n:n\in \mbN\}$
\item Is the same true for a Cauchy sequence in an incomplete metric space?
\eenu
\newpage


\itep 30\\
A sequence $(x_n)$ in $\mbR$ \textbf{increases} if $n< m$ implies $x_n \leq x_m$.  It \textbf{strictly increases} if $n < m$ implies $x_n < x_m$.  It \textbf{decreases} or \textbf{strict decreases} if $n < m$ always implies $x_n \geq x_m$ or always implies $x_n > x_m$.  A sequence is \textbf{monotone} if it increases or it decreases.
\benu
\item Prove that every sequence in $\mbR$ which is monotone and bounded converges in $\mbR$
\item Prove that this monotone sequence condition is equivalent to the least upper bound property
\eenu

\newpage

\itep 31\\
Let $(x_n)$ be a sequence in $\mbR$
\benu
\item Prove that $(x_n)$ has a monotone subsequence
\item How can you deduce that any bounded sequence in $\mbR$ has a convergent subsequence?
\eenu


\end{enumerate}
\end{document}
