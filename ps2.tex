\documentclass[12pt]{amsart}
\setlength{\parskip}{.1in}
\setlength{\parindent}{0cm}
%myalterations
\usepackage{amssymb}
\usepackage{amsmath}
\usepackage[usenames,dvipsnames,svgnames,table]{xcolor}
\usepackage[colorlinks=true,urlcolor=blue,pdfborder={0 0 .5}pdfnewwindow=true]{hyperref}
\usepackage{enumitem}
%\usepackage{amsthm}
\usepackage{graphicx}
\usepackage{verbatim}
\usepackage{tabularx}
%\usepackage{arydshln,leftidx,mathtools}
\usepackage{bm}
\usepackage{tikz}
\usepackage{tikz-cd}
\usepackage{hyperref}
\usepackage{bm}
\usepackage{wasysym}


%\setlength{\dashlinedash}{.4pt}
%\setlength{\dashlinegap}{.8pt}
%\usepackage{amsthm}
\usepackage{verbatim}
%\usepackage{commath}
%My commands
%environment abbreviations
\newcommand{\benu}{\begin{enumerate}}
\newcommand{\eenu}{\end{enumerate}}
\newcommand{\bed}{\begin{description}}
\newcommand{\ed}{\end{description}}
\theoremstyle{definition}
\newtheorem{theorem}{Theorem}
\newtheorem{notation}[theorem]{Notation}
\newtheorem{exercise}[theorem]{Exercise}
\newcommand{\bex}{\begin{exercise}}
\newcommand{\ex}{\end{exercise}}

\newcommand{\pru}{{ \bfseries \textcolor{red}{Proof:} }}

%symbol definitions
\newcommand{\un}[1]{\underline{#1}}
\newcommand{\mbZ}{\mathbb{Z}}
\newcommand{\mbR}{\mathbb{R}}
\newcommand{\mbN}{\mathbb{N}}
\newcommand{\mbQ}{\mathbb{Q}}
\newcommand{\mbI}{\mathbb{I}}
\newcommand{\hra}{\hookrightarrow}
\newcommand{\tra}{\twoheadrightarrow}
\newcommand{\lra}{\leftrightarrow}
\newcommand{\ep}{\epsilon}
\newcommand{\Ra}{\Rightarrow}
\newcommand{\mb}[1]{\mathbb{#1}}
\newcommand{\mc}[1]{\mathcal{#1}}
\newcommand{\bfs}[1]{{\bfseries #1}}
\newcommand{\bs}[1]{\boldsymbol{#1}}
%Operator definitions
\DeclareMathOperator{\Irr}{Irr}
\DeclareMathOperator{\triv}{triv}
\DeclareMathOperator{\cyc}{cyc}
\DeclareMathOperator{\lcm}{lcm}
\DeclareMathOperator{\expo}{x}
\DeclareMathOperator{\ord}{o}
\DeclareMathOperator{\imm}{im}
\DeclareMathOperator{\sgn}{sgn}
\DeclareMathOperator{\Sym}{Sym}
\DeclareMathOperator{\alt}{alt}
\DeclareMathOperator{\irr}{irr}
\DeclareMathOperator{\eqt}{Equiv}
\DeclareMathOperator{\pat}{Part}
%\DeclareMathOperator{\sgn}{sgn}
%\DeclareMathOperator{\Aut}{Aut}
\DeclareMathOperator{\Gl}{Gl}
\DeclareMathOperator{\M}{M}
\DeclareMathOperator{\Id}{Id}
\DeclareMathOperator{\fixx}{Fix}
\DeclareMathOperator{\suppp}{Supp}
\DeclareMathOperator{\gl}{Gl}
\DeclareMathOperator{\id}{Id}
\DeclareMathOperator{\Aut}{Aut}
\DeclareMathOperator{\Inn}{Inn}
\DeclareMathOperator{\orb}{orb}
\DeclareMathOperator{\ii}{I}
\DeclareMathOperator{\im}{im}
\DeclareMathOperator{\Fix}{Fix}
\DeclareMathOperator{\Co}{Co}
\DeclareMathOperator{\md}{md}
\DeclareMathOperator{\qt}{qt}
\DeclareMathOperator{\ExtendedGCD}{ExtendedGCD}
\DeclareMathOperator{\Mod}{Mod}
\DeclareMathOperator{\GCD}{GCD}
\newcommand{\nms}{\negmedspace}
\newcommand{\nts}{\negthinspace}

\newcommand{\itep}{\item {\bfseries Problem}\ }
\newcommand{\gen}[1]{\langle \nts#1 \nts\rangle}
\newcommand{\quot}[2]{#1/ #2}
\newcommand{\order}[1]{\left|<\nts #1 \nts s>\right|}

%These next two commands are for making answers. 
\newcommand{\beans}{\begin{description} \item[{ \bfseries \textcolor{red}{Answer}}]\ }
\newcommand{\eans }{\end{description}}
%\newcommand{\begin{comment}ex}{{ \bfseries \textcolor{red}{Answer}}}

%To turn the answer into problem sets use replace to replace \begin{comment} with \begin{comment} and \\end{comment}  by \end{comment}.
\newcommand{\lieb}[3][{{}}]{\frac{d^#1 #2}{d\,#3^#1}}

\title{\textbf{Math 521 - Problem Set 1}}
\author{Guy Matz}
\date{\today}

\begin{document} 

%\maketitle
%\newpage % Q1

\begin{enumerate}[series=p]
\itep 1 \\
$(0, 1)$ is an open subset of $\mbR$ but not of $\mbR^2$, when we think of $\mbR$ as the $x$-axis in $\mbR^2$.  Prove this.
\\

\textbf{Definition: Openness}\\
	A set $s \subset M$ is \textbf{open in M} if for every point $p \in S$ there is an $r \in \mbR$, $r>0$, so that for every $q \in M$,
	if $d(p,q) < r$, then $q < S$
	\begin{center}
		OR
		\\
		$\forall p \in S \, \exists r > 0 (B_r^M(p) \subseteq S)$
	\end{center}
So in $\mbR^1$ the interval $(0,1)$ is open since, going to either end of the interval, it can be shown that regardless of how close to 0 or 1, there exists a point $p \in S$ such that $d(p,q) < r$ for any arbitrarily small $r$.
\\
\\
However, the interval $(0,1)$ in $\mbR^2$ is not open since a point $q$ can be found that is not arbitrarily close to $p \in S$.  The point $q = (\frac{1}{2}, \frac{1}{2})$ satisifies this condition so that $d(p,q) < r \nRightarrow q \in S$
\\
OR
\\
Is it better to say something like $B_r^\mbR \nsubseteq S$??

\newpage

\itep 2 \\
For which intervals $[a,b]$ in $\mbR$ is the intersection $[a,b] \cap \mbQ$ a clopen subset of the metric space $\mbQ$?
\\
\textbf{Definition: Openness}\\
A set $s \subset M$ is \textbf{open in M} if for every point $p \in S$ there is an $r \in \mbR$, $r>0$, so that for every $q \in M$,
if $d(p,q) < r$, then $q < S$
\\
\textbf{Definition: Closedness}\\
A closed set contains all of it limits\\
\\
Any interval with $a, b \in \mbI$ intersected with $\mbQ$ will be  both closed and open.  DO I NEED TO SAY MORE HERE?
\newpage

\itep 3\\
Prove directly from the definition of closed set that each single point is a closed subset of a metric space.  Why does this imply that a finite set of points is also a closed set?
\\
\textbf{Definition: Closedness}\\
A closed set contains all of it limits\\
\textbf{Definition: Limit}\\
A point $p \in M$ is a limit of $S$ if there exists a sequence $(p_n)$ in $S$ that converges to it.\\
\textbf{Corollary: p. 61}\\
A finite union of closed sets is closed
\\
A single point $p \in S$ can only form the sequence $(p, p, p, \dots)$ so its limit is $p$ and therefore it is closed.\\
\\
By the result above and the corollary stated above, a finite set of points is a closed set
\newpage


\itep 5\\
Prove that a set $U \subset M$ is open if and only if none of its points are limits of its complement
\\\\
\textbf{Definition: Openness}\\
A set $s \subset M$ is \textbf{open in M} if for every point $p \in S$ there is an $r \in \mbR$, $r>0$, so that for every $q \in M$,
if $d(p,q) < r$, then $q < S$
\\
\textbf{Definition: Closedness}\\
A closed set contains all of it limits\\
\textbf{Theorem}\\
Openness is dual to closedness:  the complement of an open set is closed and the complement of a closed set is open.\\
\\
$\Downarrow$ Assuming a set $U \subset M$ is open, we want to show that none of its points are limits of its complement.  If $U$ is open, then its complement, $U^c$ is closed, and so $U^c$ contains all of its limits, i.e. $U$ does not contain any limits of $U^c$\\
$\Uparrow$ Assuming none of the points $p \in U \subset M$ are limits of its complement, we want to show that $U \subset M$ is open.  Since no points $p \in U \subset M$ are limits of its complement, $U^c$, then $U^c$ contains all of its limits.  Hence $U$ is closed.
\newpage

\itep 11\\
Let $\mc{T}$ be the collection of open subsets of a metric space $M$, and $\mc{K}$ the collection of closed subsets.  Show that there is a bijection from $\mc{T}$ onto $\mc{K}$
\\\\
Since each element of $\mc{T}$ has a complement in $\mc{K}$, and vice-versa, we have a surjection.  And the complement is unique so the mapping is one-to-one, hence there is a bijection. 
\newpage

\itep 12\\
Let $M$ be the metric space with the discrete metric, or more generally a homemorph of $M$.
\benu
\item Prove that every subset of $M$ is clopen\\
Every subset $M$ is open since 
\item Prove that every function defined on $M$ is continuous.\\
The distance function, $d$, equals 1 for all but a finite number of points, specifically $|S|$.  Therefore, as shown in Problem 30 of Homework 1, $d$ is continuous.
\item Which sequences converge in $M$?
All sequences converge to 1.
\eenu

\newpage

\itep 17 \\
Assume that $f:M \rightarrow N$ is a function from one metric space to another which satisfies the following condition:  if a sequence $(p_n)$ in $M$ converges then the sequence $(f(p_n))$ in $N$ converges.  Prove that $f$ is continuous.
\\
\textbf{Definition: Convergence}\\
The sequence $(a_i(_{i \in \mbN}))$ in $M$ converges to $b in M$ if:\\
$\forall \epsilon > 0 \, \exists N \in \mbN$ such that $\forall j>N \, d(a_j,b) < \epsilon$
\\
\textbf{Definition: Sequence}\\
A sequence in $M$ is a list of elements of $M$ indexed by $\mbN, (a_i)_{i \in \mbN}$ or $(a_i)$.  A seqence is a function $f : \mbN \to M$, the metric space\\
\textbf{Definition: Metric Space}\\
A metric space is a pair $(M, d)$, where $M$ is a set and $d$ is a distance function on pairs of $M$, so that $d : M^2 \to \mbR$ with the following:
\benu
\item Positive Definiteness: $d(x, y) > 0$, and $d(x, y) = 0 \implies x = y$
\item Symmetry: $d(x, y) = d(y, x)$
\item Triangle Inequality: $d(x, z) \leq d(x,y) + d(y,z)$
\eenu
\textbf{Theorem 2}
A function $f: M \to N$ is continuous if whenever $(a_i)$ is a sequence in $M$ and $(a_i) \to b$ in $M$, then $f((a_i)) \to f(b)$ in $N$\\
\\
Let $(p_n) \to \alpha, f((p_n)) \to \beta$.  We want to show that $f(\alpha) = \beta$.  $(p_1, \alpha, p_2, \alpha,p_3, \alpha,p_4, \alpha, \dots) \to \alpha$ and $(f(p_1), f(\alpha), f(p_2), f(\alpha), \dots) \to \beta$.  So by Theorem 2 above, $f$ is continuous.
\newpage

\itep 18\\
The simplest type of mapping from one metric space to another is an $\textbf{isometry}$.  It is a bijection $f:M \rightarrow N$ that preserves distance in the sense that for all $p,q \in M$,
$$d_N(fp, fq) = d_M(p,q).$$
If there exists an isometry from $M$ to $N$ then $M$ and $N$ are said to be isometric, $M \equiv N$.  You might have two copies of a unit equilateral triangle, one centered at the origin and one centered elsewhere.  They are isometric.  Isometric metric spaces are indistinguishable as metric spaces.
\\
\textbf{Definition: Homeomorphism}\\
If $f:M \to N$ is a bijection and $f$ is continuous and the inverse bijection $f^{-1}: N \to M$ is also continous, then $f$ is a homeomorphism ($M \cong N$)
\\
\textbf{Definition: Continuity}\\
A function $f: M \to N$ is continuous if it satisfies the $\epsilon , \delta$ condition: $\forall \epsilon > 0$ and $\forall p \in M \, \exists \delta > 0$ such that:\\
$$ q \in M \text{ and } d(p,q) < \delta \implies d(fp, fq) < \epsilon$$
\textbf{Definition: Metric Space}\\
A metric space is a pair $(M, d)$, where $M$ is a set and $d$ is a distance function on pairs of $M$, so that $d : M^2 \to \mbR$ with the following:
\\
\benu
\item Prove that every isometry is continuous
The isometric function $f: M \to N$ satisfies the $\epsilon , \delta$ condition since, by definition $d(p, q) = d(fp, fq)$, so $\epsilon = \delta = 0$
\item Prove that every isometry is a homeomorphism
By definition, $f$ is bijective.  By 18a above, $f$ is continuous as is$f^{-1}$ since $f(fp, fq) = d(f^{-1}p, f{-1}q) = d(p,q)$.  Hence every isometry is a homeomorphism
\item Prove that $[0,1]$ is not isometric to $[0,2]$.\\
$[0,2]$ has a bigger distance than $[0,1]$\hfill\lightning
\\Is this enough?
\eenu
\newpage
\itep 19\\
Prove that isometry is an equivalence relation: if $M$ is isometric to $N$, show that $N$ is isometric to $M$; show that any $M$ is isometric to itself (what mapping of $M$ to $N$ is isometry?); if $M$ is isometric to $N$ and $N$ is isometric to $P$, show that $M$ is isometric to $P$.
\benu
\item $M \equiv M$\\
$d_M(p,q) = d_M(p,q)$
\item $M \equiv N \implies N \equiv M$\\
If $M \equiv N$, then $d_M(p,q) = d_N(fp, fq)$ and $d_N(f^{-1}p, f^{-1}q) = d_M(p,q)$, so $N \equiv M$
\item $M \equiv N \text{ and }  N \equiv M \implies M \equiv P$\\
Since $M \equiv N$ and $N \equiv P$ there exist functions $f$ and $g$ such that $d_M(p,q) = d_N(fp, fq)$ and $d_N(fp, fq) = d_P(gfp, gfq)$.  So $d_M(p,q) =  d_P(gfp, gfq)$, hence $M \equiv P$
\eenu
\newpage

\itep 20\\
Is the perimeter of a square isometric to the circle?  Homeomorphic?  Explain.\\
No.  In $\mbR^2$, the corners of the unit square are $sqrt{2}$ from the diagonal corner, while all other points are 1 unit away from their opposite side.\\
\\
The two are homeomorphic, though.  I.e, Continuous in $f$ and inverse in both ways.  Any point drawn outward from the from the center of the circle to the square is unique and all points on the square are "hit" so it is bijective, as is its inverse.
\newpage

\itep 22\\
Is $\mbR$ homemorphic to $\mbQ$?  Explain.\\
No.  $|\mbQ| < |\mbR|$ so there can be no function $f$ that is bijective, since no $f$ can be "onto" and one-to-one.
\newpage

\itep 23\\
Is $\mbQ$ homemorphic to $\mbN$?  Explain.\\
No.  A mapping to $\mbN$ cannot be continuous
\newpage

\itep 29\\
\benu
\item Prove that every convergent sequence is bounded.  That is, if $(p_n)$ converges in the metric space $M$, prove that there is some neighborhood $M_rq$ containing the set $\{p_n:n\in \mbN\}$
\item Is the same true for a Cauchy sequence in an incomplete metric space?
\eenu
\textbf{Definition: The r-neighborhood of p}:\\
$M_rq = \{q \in M: d(p,q) < r\}$
\\
\textbf{Definition: Incomplete Metric Space}:\\
A metric space $M$ is called complete (or a Cauchy space) if every Cauchy sequence of points in $M$ has a limit that is also in $M$ or, alternatively, if every Cauchy sequence in $M$ converges in $M$.
\newpage


\itep 30\\
A sequence $(x_n)$ in $\mbR$ \textbf{increases} if $n< m$ implies $x_n \leq x_m$.  It \textbf{strictly increases} if $n < m$ implies $x_n < x_m$.  It \textbf{decreases} or \textbf{strict decreases} if $n < m$ always implies $x_n \geq x_m$ or always implies $x_n > x_m$.  A sequence is \textbf{monotone} if it increases or it decreases.
\benu
\item Prove that every sequence in $\mbR$ which is monotone and bounded converges in $\mbR$
\item Prove that this monotone sequence condition is equivalent to the least upper bound property
\\\\
A \textbf{Intuition}: Monotone + bounded means that it approaches a least upper bound
\\
B \textbf{Intuition}: I think I just did
\eenu

\newpage

\itep 31\\
Let $(x_n)$ be a sequence in $\mbR$
\benu
\item Prove that $(x_n)$ has a monotone subsequence
\item How can you deduce that any bounded sequence in $\mbR$ has a convergent subsequence?
\eenu
\textbf{Definition: Subsequence}\\
If $(p_n)_{n \in \mbN}$ and $(q_k)_{k \in \mbN}$ ar sequences and if there is a sequence  $1 \leq n_1 < n_2 < n_3 < \dots$ of integers such that for each $k \in \mbN, q_k = q_{n_k}$ then $(q_k)$ is a \textbf{subsequence} of $(p_n)$
\\
\\
A \textbf{Intuition}: Since a subsequence can be chosen arbitrarily, simply choose the elements from the parent sequence that are either increasing or decreasing
\\
\\
B \textbf{Intuition}: Simply choose the elements from the parent sequence to create a monotone subsequence.  By 30a above, a bounded, monotone subsequence converges.
\end{enumerate}
\end{document}
