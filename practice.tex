\documentclass[12pt]{amsart}
\setlength{\parskip}{.1in}
\setlength{\parindent}{0cm}
%myalterations
\usepackage{amssymb}
\usepackage{amsmath}
\usepackage[usenames,dvipsnames,svgnames,table]{xcolor}
\usepackage[colorlinks=true,urlcolor=blue,pdfborder={0 0 .5}pdfnewwindow=true]{hyperref}
\usepackage{enumitem}
%\usepackage{amsthm}
\usepackage{graphicx}
\usepackage{verbatim}
\usepackage{tabularx}
%\usepackage{arydshln,leftidx,mathtools}
\usepackage{bm}
\usepackage{tikz}
\usepackage{tikz-cd}
\usepackage{hyperref}
\usepackage{bm}

%\setlength{\dashlinedash}{.4pt}
%\setlength{\dashlinegap}{.8pt}
%\usepackage{amsthm}
\usepackage{verbatim}
%\usepackage{commath}
%My commands
%environment abbreviations
\newcommand{\bem}{\begin{matrix}}
\newcommand{\emm}{\end{matrix}}
\newcommand{\besm}{\begin{smallmatrix}}
\newcommand{\esm}{\end{smallmatrix}}
\newcommand{\benu}{\begin{enumerate}}
\newcommand{\eenu}{\end{enumerate}}
\newcommand{\bed}{\begin{description}}
\newcommand{\ed}{\end{description}}
\theoremstyle{definition}
\newtheorem{theorem}{Theorem}
\newtheorem{notation}[theorem]{Notation}
\newcommand{\bnot}{\begin{notation}}
\newcommand{\enot}{\end{notation}}
\newcommand{\bet}{\begin{theorem}}
\newcommand{\et}{\end{theorem}}
\newtheorem{axiom}[theorem]{Axiom}
\newcommand{\baxi}{\begin{axiom}}
\newcommand{\axi}{\end{axiom}}
\newtheorem{lemma}[theorem]{Lemma}
\newcommand{\bel}{\begin{lemma}}
\newcommand{\el}{\end{lemma}}
\newtheorem{corollary}[theorem]{Corollary}
\newcommand{\bec}{\begin{corollary}}
\newcommand{\ec}{\end{corollary}}
\newtheorem{observation}[theorem]{Observation}
\newcommand{\bo}{\begin{observation}}
\newcommand{\eo}{\end{observation}}
\newtheorem{exercise}[theorem]{Exercise}
\newcommand{\bex}{\begin{exercise}}
\newcommand{\ex}{\end{exercise}}

\newtheorem{definition}[theorem]{Definition}
\newcommand{\bdf}{\begin{definition}}
\newcommand{\edf}{\end{definition}}
\newtheorem{example}[theorem]{Example}
\newcommand{\bax}{\begin{example}}
\newcommand{\ax}{\end{example}}
\newcommand{\pru}{{ \bfseries \textcolor{red}{Proof:} }}

\newtheorem*{und}{Definition}
%symbol definitions
\newcommand{\un}[1]{\underline{#1}}
\newcommand{\mbZ}{\mathbb{Z}}
\newcommand{\mbR}{\mathbb{R}}
\newcommand{\mbN}{\mathbb{N}}
\newcommand{\mbQ}{\mathbb{Q}}
\newcommand{\mbC}{\mathbb{C}}
\newcommand{\mbF}{\mathbb{F}}
\newcommand{\mcS}{\mathcal{S}}
\newcommand{\mcP}{\mathcal{P}}
\newcommand{\hra}{\hookrightarrow}
\newcommand{\tra}{\twoheadrightarrow}
\newcommand{\lra}{\leftrightarrow}
\newcommand{\ep}{\epsilon}
\newcommand{\Ra}{\Rightarrow}
\newcommand{\mb}[1]{\mathbb{#1}}
\newcommand{\mc}[1]{\mathcal{#1}}
\newcommand{\bfs}[1]{{\bfseries #1}}
\newcommand{\bs}[1]{\boldsymbol{#1}}
%Operator definitions
\DeclareMathOperator{\Irr}{Irr}
\DeclareMathOperator{\triv}{triv}
\DeclareMathOperator{\cyc}{cyc}
\DeclareMathOperator{\lcm}{lcm}
\DeclareMathOperator{\expo}{x}
\DeclareMathOperator{\ord}{o}
\DeclareMathOperator{\imm}{im}
\DeclareMathOperator{\sgn}{sgn}
\DeclareMathOperator{\Sym}{Sym}
\DeclareMathOperator{\alt}{alt}
\DeclareMathOperator{\irr}{irr}
\DeclareMathOperator{\eqt}{Equiv}
\DeclareMathOperator{\pat}{Part}
%\DeclareMathOperator{\sgn}{sgn}
%\DeclareMathOperator{\Aut}{Aut}
\DeclareMathOperator{\Gl}{Gl}
\DeclareMathOperator{\M}{M}
\DeclareMathOperator{\Id}{Id}
\DeclareMathOperator{\fixx}{Fix}
\DeclareMathOperator{\suppp}{Supp}
\DeclareMathOperator{\gl}{Gl}
\DeclareMathOperator{\id}{Id}
\DeclareMathOperator{\Aut}{Aut}
\DeclareMathOperator{\Inn}{Inn}
\DeclareMathOperator{\orb}{orb}
\DeclareMathOperator{\ii}{I}
\DeclareMathOperator{\im}{im}
\DeclareMathOperator{\Fix}{Fix}
\DeclareMathOperator{\Co}{Co}
\DeclareMathOperator{\md}{md}
\DeclareMathOperator{\qt}{qt}
\DeclareMathOperator{\ExtendedGCD}{ExtendedGCD}
\DeclareMathOperator{\Mod}{Mod}
\DeclareMathOperator{\GCD}{GCD}
\newcommand{\nms}{\negmedspace}
\newcommand{\nts}{\negthinspace}


\newcommand{\itep}{\item {\bfseries Problem}\ }
\newcommand{\gen}[1]{\langle \nts#1 \nts\rangle}
\newcommand{\quot}[2]{#1/ #2}
\newcommand{\order}[1]{\left|<\nts #1 \nts s>\right|}

%These next two commands are for making answers. 
\newcommand{\beans}{\begin{description} \item[{ \bfseries \textcolor{red}{Answer}}]\ }
\newcommand{\eans }{\end{description}}
%\newcommand{\begin{comment}ex}{{ \bfseries \textcolor{red}{Answer}}}

%To turn the answer into problem sets use replace to replace \begin{comment} with \begin{comment} and \\end{comment}  by \end{comment}.
\newcommand{\lieb}[3][{{}}]{\frac{d^#1 #2}{d\,#3^#1}}

\title{\textbf{Math 311 - Final}}
\author{Guy Matz}
\date{\today}

\begin{document} 

\maketitle
%\newpage % Q1

\begin{enumerate}[series=p]
\itep 
\label{drp}
Show that $\sum_{i=1}^{n} i = \dfrac{n(n+1)}{2}$
\\
Proof by Induction on $n$:
\\
Base Step: 
$$\sum_{i=1}^{1} = 1$$
Induction Hypothesis:
$$\sum_{i=1}^{n} i = \dfrac{n(n+1)}{2}$$
Proof:
$$\sum_{i=1}^{n+1} i = \dfrac{(n+1)(n+2)}{2} = \dfrac{n^2 + 3n + 2}{2}$$
and $$\sum_{i=1}^{n+1} i = \sum_{i=1}^{n} i + (n+1)$$
By our Induction Hypothesis, this is equal to:
\\
$$\dfrac{n(n+1)}{2} + (n+1) = \dfrac{n^2 + 3n + 2}{2}$$
\newpage

\itep
Show that the interior angles of a triangle add to 180

Let us draw a line perpendicular to the base of the triangle, and through the apex.  We shall call this line $l$.  At the point where $l$ intersects the two sides of the triangle we now have three angles, $a, b$ and $c$ in the diagram below.  The sides of the triangles act as intersecting transversals, and so the angles $d$ and $f$ in the diagram below are congruent to angles $a$ and $c$. Therefore the inner angles of the triangle add up to the angle of the line $l$, which is known to be 180
\\
\begin{tikzpicture}
\draw (0,0) node[above right=1pt+1pt]{$d$}
-- (4,0) node[above left=5pt]{$f$}
-- (3,4) node[below left=5pt]{$a$} node[below=5pt]{$b$} node[below right=5pt]{$c$}
-- cycle;
\draw[red] (-1,0) -- (5,0);
\draw[red] (-1,4) -- (5,4);
\end{tikzpicture}
\newpage

\itep
Show that there are infinitely many primes
\\

\newpage

\itep
Show that $2^n \geq n$ for all $n \geq 0$
\\
Proof by Induction on $n$.\\
Base Step: $n$ = 1
$$2^1 \geq 1$$
\\
Induction Hypothesis:\\
$$2^n \geq n$$
Induction Step:\\
\begin{align*}
2^{n+1} &\geq  n+1\\
2 \times 2^n &\geq n + 1
\end{align*}
By our Induction Hypothesis we have that $2^n \geq n$,
so do we known that $2 \times 2^n \geq n+1$?  Do natural logs help us any here?
\newpage

\itep
Show that if $e, g \in G$ are identity elements, then $e = g$.
\\
Let $a \in G$, then:\\
$ ea = a$ and $ga = a$ so $ea = ga$, hence $e = g$
\end{enumerate}
\end{document}
